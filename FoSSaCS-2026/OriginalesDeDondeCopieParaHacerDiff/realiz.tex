\section{Realizability model}\label{sec:model}

In this section, we present the type system corresponding to the untyped language introduced in the previous section, along with its realizability semantics.

\subsection{Unitary Type Semantics}

Given the deterministic machine presented in the previous section, the next step to extract a typing system is to define the sets of values which will characterize its types. In order to achieve this we first need to identify the notion of what exactly constitutes a type.

Our aim is to define types that are exclusively inhabited by values of norm equal to 1. The vectors that we wish to study all fall in the \emph{unit sphere}. We will write $\Sph$ for the set $\Sph := \{\vec v \in \vec{\Val}~|~\|\vec v\| =1\}$. This corresponds with the mathematical notion of representing quantum data as unit vectors in a Hilbert space. 

\begin{definition}[Unitary value distribution]
  We say a value distribution $\vec{v}$ is unitary when it has norm equal to $1$. In other words, when $\vec{v}\in\Sph$.
\end{definition}

\begin{definition}[Unitary type]
  We define a \textit{unitary type} (or just \textit{type}) as a notation $A$ together with a set of unitary value distributions noted $\sem{A}$ called the unitary semantics of $A$.
\end{definition}

We next move onto the type realizers. Since our aim is to extract a quantum lambda calculus, we wish to filter global phases of qubits at this level. Since the global phase of a quantum state has no physical significance, we wish to assign the same types to a term $\vec{t}$ and $e^{i\theta}\cdot\vec{t}$. This idea will guide the definition of type realizers.

\begin{definition}[Type realizer]
  Given a type $A$ and a term distribution $\vec t$, we say that $\vec t$ realizes $A$ (noted $\vec t \real A$), when there is a value distribution $\vec v$ such that:
  \begin{itemize}
    \item $\vec{t}\twoheadrightarrow e^{i\theta}\cdot\vec{v}$
    \item $\vec{v}\in\sem{A}$
  \end{itemize}
  For each type $A$, we note the set of its realizers as $\{\real A\}$.
\end{definition}

With the notions of unitary types and its realizers we can start defining the specific approach for our previously defined language. We begin with the type grammar defined on \Cref{tab:UnitaryTypes} and build a simple algebra from the sets of values we aim to represent. From this point onwards denote by $\Type$ the set of all types and by $\BasisType$ the set of all bases.

\begin{table*}[tb]
  \scriptsize
    \[
    T := \basis{X} \mid T\to T \mid T\times T \mid \sharp T
    \]
    \begin{align*}
    \sem{\basis{X}}&:= X\qquad\text{Where: $X$ is an orthonormal basis}\\
    \sem{A\times B}&:= \bigl\{ (\vec v, \vec w): \vec v \in{\sem{A}},~\vec w\in\sem{B}\bigr\}\\
    \sem{A\Arr B}&:=
    \bigl\{\sum_{i=1}^{n}\alpha_i(\Lam{x}{B}{\vec{t_i}})\in\Sph:\forall\vec{w}\in\sem{A}, (\sum_{i=1}^{n}\alpha_i \vec{t_i})\ansubst{\vec{w}/x}{A}\real B\bigr\}\\
    \sem{\sharp{A}}&:= {(\sem{A}^\bot)}^\bot\\
    &\text{Where: }\comp{A} = \{ \vec{v}\in \Sph \,\mid\, \scal{\vec{v}}{a} = 0,\, \forall a\in A\}
  \end{align*}
  \caption{Type notations and semantics}
  \label{tab:UnitaryTypes}
\end{table*}

The types $\basis{X}$ act as atomic types. They represent a finite set $X$ of orthogonal vectors forming an orthonormal basis. We can represent boolean values with a basis of size 2, but we are not limited to only one kind since there are infinite bases to choose from.

The type $A\times B$ represents the cartesian product of $A$ and $B$. However, the syntax grammar only allows for pairs of pure values. So there is a small subtlety on the type depicted in the table. For every $\vec{v}=\sum_{i=1}^{n}\alpha_i v_i\in\sem{A}$ and $\vec{w}=\sum_{j=1}^{m}\beta_j w_j\in\sem{B}$ (With $v_i$ and $w_j$ pure values) when we filter out the notation for pairs, we get:

\[
  \sem{A\times B}:= \bigl\{ \sum_{i=1}^{n}\sum_{j=1}^{m}\alpha_i\beta_j(v_i, w_j): \vec v \in{\sem{A}},~\vec w\in\sem{B}\bigr\}
\]

We stress this fact for rigorousness, but for ease of reading from this point onwards we will make use of the previously defined notation.

The arrow type $A\Arr B$ is composed by the distributions of lambda abstractions that take values from the interpretation of $A$ to realizers of $B$. The last type $\sharp A$ takes the double orthogonal complement and intersects it with the unit sphere. 

The type grammar is standard except for type $\sharp A$. We use it to represent quantum data, i.e. linear resources, so terms of this type will not be able to be erased or duplicated. This can be thought as the opposite of the \textit{bang} ($!$) modality in linear logic. For a more in-depth analysis, refer to \cite{DiazcaroCIE2025}.

Intuitively, applying the sharp ($\sharp$) operator to a type $A$ yields the span of the original type (intersected with the unitary sphere). This describes the possible linear combinations of values in the unitary semantics of $A$. The following proposition proves that characterization:

\begin{proposition}\label{prop:SharpCharacterization}
  The type interpretation $\sem{\sharp A}$ contains the norm-$1$ linear combination of values in $\sem{A}$.
  \[
  \sem{\sharp A} = (\sem{A}^\bot)^\bot = \Span(\sem{A})\cap\Sph
  \]
\end{proposition}

\begin{proof}
  Proof by double inclusion.
  \begin{description}
    \item[$\Span(\sem{A})\cap\Sph\subseteq (\sem{A}^\bot)^\bot$:] Let $\vec{v}\in\Span(\sem{A})\cap\Sph$. Then $\vec{v}$ is of the form $\sum_{i=1}^{n}\alpha_i \vec{v_i}$ with $\vec{v_i}\in\sem{A}$. Taking $\vec{w}\in\sem{A}^\bot$, we examine the inner product:
    
    \begin{align*}
    \scal{\vec{v}}{\vec{w}} &= \scal{\sum_{i=1}^{n}\alpha_i \vec{v_i}}{\vec{w}}\\
    &= \sum_{i=1}^{n}\overline{\alpha_i}\scal{\vec{v_i}}{\vec{w}}=0
    \end{align*}

    Then $\vec{v}\in(\sem{A}^\bot)^\bot$.

    \item[$(\sem{A}^\bot)^\bot\subseteq \Span(\sem{A})\cap\Sph$:] Reasoning by contradiction, we assume that there is a $\vec{v}\in(\sem{A}^\bot)^\bot$ such that $v\not\in\Span(\sem{A})\cap\Sph$. Since $\vec{v}\not\in\Span(\sem{A})$, $\vec{v}=\vec{w_1} + \vec{w_2}$ such that $\vec{w_1}\in\Span{\sem{A}}$ and $\vec{w_2}$ is a non-null vector which cannot be written as a linear combination of elements of $\sem{A}$. In other words, $\vec{w_2}\in\sem{A}^\bot$. Taking the inner product:
    \[
    \scal{\vec{v}}{\vec{w_2}} = \scal{\vec{w_1}+\vec{w_2}}{\vec{w_2}} = \|\vec{w_2}\|\neq 0
    \]
    Then $\vec{v}\not\in(\sem{A}^\bot)^\bot$. The contradiction stems from assuming $\vec{v}\not\in\Span{\sem{A}}\cap\Sph$.\qedhere
  \end{description}
\end{proof}

The following proposition shows that, as one would expect from the span, multiple applications of the sharp operator does not produce a different result beyond the first one.   

\begin{proposition}\label{prop:IdempotentSharp}
  The $\sharp$ operator is idempotent, that is $\sem{\sharp A} = \sem{\sharp (\sharp A)}$
\end{proposition}

\begin{proof}
  We want to prove that $(((\comp{\sem{A}})^\bot)^\bot)^\bot = (\comp{\sem{A}})^\bot$. For ease of reading, we will write $\comp[n]{A}$ for $n$ successive applications of the operation $\bot$.

  \begin{description}
    \item[$A\subseteq A^{\bot^2}$:] Let $\vec{v}\in A$. Then, for all $\vec{w}\in\comp{A}$, $\scal{\vec{v}}{\vec{w}} = 0$. Then $\vec{v}\in\comp[2]{A}$. With this we have $A\subseteq\comp[2]{A}$.
    
    \item[$A^{\bot^3}\subseteq \comp{A}$:] Let $\vec{u}\in \comp[3]{A}$. Then, for all $\vec{v}\in\comp[2]{A}$, $\scal{\vec u}{\vec v} = 0$. Since we have shown that $A\subseteq \comp[2]{A}$, we have that for all $\vec{w}\in A$, $\scal{\vec u}{\vec w} = 0$. Then $\vec u\in\comp{A}$. With this we have $\comp[3]{A}\subseteq \comp{A}$.
  \end{description}

  With these two inclusions we have that $\comp{A}=\comp[3]{A}$. So we conclude that: $\sem{\sharp(\sharp A)} = \comp[4]{A} = \comp[2]{A} = \sem{\sharp A}$ \qedhere
\end{proof}

\begin{remark}
  A basis type $\basis{X}$ may be formed by value distributions of pairs and so might be written as the product type of smaller bases. For example, let $X=\{\ket{00}, \ket{01}, \ket{10}, \ket{11}\}$, then $\basis{X}=\B\times\B$. However, for the case of entangled bases this cannot be done. A clear example is the Bell basis: $\mathsf{Bell}=\{\frac{\ket{00}+\ket{11}}{\sqrt{2}},\frac{\ket{00}-\ket{11}}{\sqrt{2}},\frac{\ket{01}+\ket{10}}{\sqrt{2}},\frac{\ket{01}-\ket{10}}{\sqrt{2}}\}$.
\end{remark}

The only thing left would be to check that our type algebra captures sets of value distributions we wish to study. \Cref{prop:UnitaryTypes} states that every member of a type interpretation has norm $1$.

\begin{proposition}\label{prop:UnitaryTypes}
  For every type $A$, $\sem{A}\subseteq\Sph$.
\end{proposition}

\begin{proof}
  Proof by induction on the shape of $A$. Since by definition, $\sem{\basis{X}}$, $\sem{A\Arr B}$ and $\sem{\sharp{A}}$ are built from values in $\Sph$ the only case we need to examine is $\sem{A\times B}$.
  
  Let $\vec v = \sum_{i=0}^{n} \alpha_i v_i \in\sem{A}$ and $\vec w = \sum_{j=0}^{m} \beta_j w_j$ where every $v_i$ are pairwise orthogonal, same for $w_j$. Then:
     
  \[(\vec v, \vec w) = \sum_{i=0}^{n} \sum_{j=0}^{m} \alpha_i\beta_j (v_i,w_j)\]
  
  So we have: 
  \[\|\Pair{\vec v}{\vec w}\| = \sqrt{\sum_{i=1}^n\sum_{j=1}^{m} |\alpha_i\beta_j|^2} = \sqrt{\sum_{i=1}^n |\alpha_i|^2 \sum_{j=1}^{m} |\beta_j|^2}\]

  Since both $\vec v\in\sem{A}$ and $\vec w\in\sem{B}$, by inductive hypothesis, we have that $\|\vec v\| = \| \vec w \| = 1$. Which is to say $\sum_{i=1}^{n} |\alpha_i|^2 = \sum_{j=1}^{m} |\beta_j| = 1$. So we conclude $\|\Pair{\vec{v}}{\vec{w}}\| = 1$.
  
\end{proof}

Defining types as sets of values also induces an intuitive way to define a subtyping relationship. We say a type $A$ is subtype of a type $B$ (noted $A\leq B$) if the set of realizers of $A$ is included in the set of realizers of $B$ ($\{\real A\}\subseteq\{\real B\}$). If the sets coincide, we say that $A$ is isomorphic to $B$ (noted $A\cong B$). 

\begin{example}
  For example, for every type $A$, $A\leq\sharp A$. For bases, $\basis{\B}$ and $\basis{\XB}$ we have that: neither $\basis{\B}\not\leq\basis{\XB}$, nor $\basis{\B}\not\leq\basis{\XB}$. However, $\sharp\basis{\B}\cong\sharp\basis{\XB}$.
\end{example}

Although every type is defined by norm 1 value distributions, not every norm 1 distribution belongs to the interpretation of a type. Take for example the distribution $\frac{1}{\sqrt{2}} (\ket{0} + \Pair{\ket{0}}{\ket 0})$. Another case is a linear combination of abstractions with different bases. For example, the term:

\[
\frac{1}{\sqrt{2}}(\Lam{x}{\B}{\pauliX{x}}) + \frac{1}{\sqrt{2}}(\Lam{x}{\XB}{x})
\]

Is not a member of an arrow type, since the bases decorating each abstraction do not match. However, it is computationally equivalent to the abstraction $(\Lam{x}{\B}{\ket{+}})$ which belongs to the set $\sem{\basis{\B}\Arr\basis{\XB}}$.

\subsection{Characterization of unitary operators}


One of the main results of \cite{DiazcaroGuillermoMiquelValironLICS19}, is the characterization of $\C^2\to\C^2$ unitary operators using values in $\sem{\sharp\B\Arr\sharp\B}$ \cite[Theorem IV.12]{DiazcaroGuillermoMiquelValironLICS19}. In this subsection we expand on this result. Our goal is to prove that abstractions of type $\sharp\basis{X}\Arr\sharp\basis{Y}$ (both bases of size $n$) represent $\C^n\to\C^n$ unitary operators.

Unitary operators are the isomorphisms of Hilbert spaces since they preserve the basic structure of the space. With this in mind, the first step is to show that the members in $\sharp\basis{X}\Arr\sharp{\basis{Y}}$ send basis vectors from $\basis{X}$ onto orthogonal vectors in $\sem{\sharp\basis{Y}}$. In other words, these abstractions preserve both norm and orthogonality.

\begin{lemma}\label{lem:BasesIso}
  Given types $\basis{X}$, $\basis{Y}$ of size $n$ and a closed $\lambda$-abstraction $\Lam{x}{X}{\vec t}$ we have that $\Lam{x}{A}{\vec t}\in\sem{\sharp\basis{X}\Arr\sharp\basis{Y}}$ if and only if there are value distributions $\vec w_i\in\sem{\sharp\basis{Y}}$ such that $\forall\vec v_i\in\sem{\basis{X}}$:
  \[
    \vec{t}[\vec{v_i}/x]\eval\vec{w_i}\perp\vec{w_j}\twoheadleftarrow \vec{t}[\vec{v_j}/x] \qquad \text{if } i\neq j
  \]
\end{lemma}

\begin{proof}
  \textit{The condition is necessary:} Suppose that $\Lam{x}{X}{\vec t_k}\in\sem{\sharp\basis{X}\Arr\sharp\basis{Y}}$, thus $\forall \vec v_i\in\sem{\sharp\basis{X}},\ \vec{t}\ansubst{\vec{v_i}/x}{X}\eval\vec w_i\in\sem{\sharp\basis{Y}}$. It remains to be seen that $\vec w_i \perp \vec w_j$ if $i\neq j$. For that, we consider $\alpha_i\in\C$ such that $\sum_{i=1}^n |\alpha_i|^2 = 1$. By linear application on the basis $X$ we observe that:
  \begin{align*}
    (\Lam{x}{X}{\vec{t}})(\sum_{i=1}^n \alpha_i \vec{v_i}) &\to \vec t\ansubst{\sum_{i=1}^n \alpha_i \vec v_i/x}{X}\\
    &= \sum_{i=1}^{n} \alpha_i \vec{t}[\vec{v_i}/x]\\ 
    &\twoheadrightarrow \sum_{i=1}^n \alpha_i \vec w_i
  \end{align*}

  But since $\sum_{i=1}^n \alpha_i \vec{v_i}\in\sem{\sharp A}$, then $\sum_{i=1}^n \alpha_i \vec{w_i}\in\sem{\sharp B}$ too. Which implies $\|\sum_{i=1}^n \alpha_i \vec{w_i}\|=1$. Therefore:
  \begin{align*}
    1 = \|\sum_{i=1}^n \alpha_i \vec{w_i}\| &= \scal{\sum_{i=1}^n \alpha_i \vec{w_i}}{\sum_{j=1}^n \alpha_j \vec{w_j}}\\
    &=\sum_{i=1}^n |\alpha_i|^2 \scal{\vec w_i}{\vec w_i } + \sum_{i,j=1; i\neq j}^n \bar{\alpha_i}\alpha_j \scal{\vec w_i}{\vec w_j}\\
    &=\sum_{i=1}^n |\alpha_i|^2 \scal{\vec w_i}{\vec w_i } + \sum_{i,j=1; i<j}^n 2~\Rpart{\bar{\alpha_i}\alpha_j \scal{\vec w_i}{\vec w_j}}\\
    &=\sum_{i=1}^n |\alpha_i|^2 \|\vec w_i\|^2 + 2\sum_{i,j=1; i<j}^n \Rpart{\bar{\alpha_i}\alpha_j \scal{\vec w_i}{\vec w_j}}\\
    &=\sum_{i=1}^n |\alpha_i|^2 + 2\sum_{i,j=1; i<j}^n\Rpart{\bar{\alpha_i}\alpha_j \scal{\vec w_i}{\vec w_j}}\\
    &= 1 + 2\sum_{i,j=1; i<j}^n \Rpart{\bar{\alpha_i}\alpha_j \scal{\vec w_i}{\vec w_j}}
  \end{align*}

  And thus we are left with $\sum_{i,j=1; i<j}^n \Rpart{\bar{\alpha_i}\alpha_j \scal{\vec w_i}{\vec w_j}} = 0$. Taking $\alpha_{i'} = \alpha_{j'} = \frac{1}{\sqrt{2}}$ with $0$ for the rest of coefficients, we have $\Rpart{\scal{\vec w_{i'}}{\vec w_{j'}}} = 0$ for any two arbitrary $i'$ and $j'$. In the same way, taking $\alpha_{i'} = \frac{1}{\sqrt{2}}$ and $\alpha_{j'}=\frac{i}{\sqrt{2}}$ with $0$ for the rest of the coefficients, we have $\Ipart{\scal{\vec{w_{i'}}}{\vec w_{j'}}} = 0$ for any two arbitrary $i'$ and $j'$. Finally, we can conclude that $\scal{\vec w_i}{\vec w_j}=0$ if $i\neq j$.

  \textit{The condition is sufficient:} Suppose that there are $\vec{w_i}\in\sem{\sharp\basis{Y}}$ such that for every $\vec v_i\in\sem{\basis{X}}$:
  \[
    \vec t[\vec v_i/x] \eval \vec{w_i} \perp \vec{w_j} \twoheadleftarrow \vec t[\vec v_j/x]\qquad \text{If } i\neq j
  \]
  Given any $\vec u\in\sem{\sharp\basis{X}}$ we have that $\vec u = \sum_{i=1}^n \alpha_i \vec v_i$ with $\sum_{i=1}^n |\alpha_i|^2 = 1$ and $\vec v_i\in\sem{\basis{X}}$. Then 
  \[
    (\Lam{x}{X}{\vec t}) \vec u \to \vec t_k\ansubst{\vec u/x}{X}=\sum_{i=1}^{n}\alpha_i \vec{t}[\vec{v_i}/x]\eval\sum_{i=1}^n \alpha_i\vec w_i
  \]

  We have that for each $i$, $\vec w_i\in\sem{\sharp\basis{Y}}$. In order to show that $(\Lam{x}{A}{\vec t})\vec u\real\sharp\basis{Y}$ we still have to prove that $\|\sum_{i=1}^n \alpha_i \vec w_i\| = 1$

  \begin{align*}
    \|\sum_{i=1}^n \alpha_i \vec w_i\|^2 &= \scal{\sum_{i=1}^n \alpha_i \vec w_i}{\sum_{j=1}^n \alpha_j \vec w_j}\\
    &=\sum_{i=1}^n |\alpha_i|^2 \scal{\vec w_i}{\vec w_i } + \sum_{i,j=1; i\neq j}^n \bar{\alpha_i}\alpha_j \scal{\vec w_i}{\vec w_j}\\
    &=\sum_{i=1}^n |\alpha_i|^2 + 0\\
    &= 1
  \end{align*}

  Then $\sum_{i=1}^n \alpha_i \vec w_i\in\sem{\sharp(\sharp\basis{Y})}=\sem{\sharp\basis{Y}}$ by \Cref{prop:IdempotentSharp}. Since for every $\vec u\in\sem{\sharp A}$, $(\Lam{x}{A}{\vec t}) \vec u\real\sharp B$, we can conclude that $\Lam{x}{A}{\vec t}\in\sem{\sharp A\to\sharp B}$.\qedhere
\end{proof}

Next, we need to bridge the gap between the values in the calculus with vectors in the space $\C^n$. In order to do this, we introduce a meta-language operation $\pi_n$ which translates value distributions into vectors in $\C^n$. The operation simply writes the value in the canonical basis and takes the corresponding coefficients. 

\begin{definition}
Let $\basis{X}$ be an orthonormal basis of size $n$, then for every $\vec{v}\in\sem{\basis{X}}$:
\[
\vec{v}\equiv \sum_{i=1}^{n}\alpha_i\ket{i}
\]
Where $\ket{i}$ is the $n$-th dimensional product of $\ket{0}$ and $\ket{1}$ with $i$ written in binary and $\sum_{i=1}^{n}|\alpha_i|^2=1$. (For example, $\ket{3}$ with $n=4$ is $\ket{0011}$). We define $\pi_n:\sem{\basis{X}}\to\C^n$ as::
\[
\pi_n(\vec{v}) = (\alpha_1,\dotsb ,\alpha_n)
\]
We will omit the subscript when it can be deduced from the context.
\end{definition}

\begin{definition}
We say a $\lambda$-abstraction $(\Lam{x}{X}{\vec{t}})$ represents an operator $F:\C^n\to\C^n$ when:
\[
(\Lam{x}{X}{\vec{t}})\vec{v} \eval \vec{w} \iff F(\pi_n(\vec{v})) = \pi_n(\vec{w})
\]
\end{definition}

This means, a lambda term represents a function $F:\C^n\to\C^n$ if it encodes the action of $F$ on vectors. This definition, in conjunction with the previous lemma, allow us build a characterization of unitary operators as values in $\sharp\basis{X}\Arr\sharp\basis{X}$.

\begin{theorem}
  Let $\basis{X}$, $\basis{Y}$ be orthonormal bases of size $n$. A closed $\lambda$-abstraction $(\Lam{x}{X}{\vec{t}})\in\sem{\sharp\basis{X}\Arr\sharp\basis{Y}}$ if and only if it represents a unitary operator $F:\C^n\to\C^n$.
\end{theorem}

\begin{proof}
  \textit{The condition is necessary:} Suppose that $(\Lam{x}{X}{\vec{t}})\in\sem{\sharp\basis{X}\Arr\sharp\basis{Y}}$, then by \Cref{lem:BasesIso} we have that, for every $\vec{v_i}\in\sem{\basis{X}}$  there exist $\vec{w_i}\in\sem{\sharp\basis{Y}}$ such that $\vec{t}[\vec{v_i}/x]\eval\vec{w_i}$ and $\vec{w_i}\perp\vec{w_j}$ if $i\neq j$. Let $F:\C^n\to\C^n$ be the operator defined as $F(\pi(\vec{v_i}))=\pi(\vec{w_i})$. From the linear application on $X$, it is clear that $(\Lam{x}{X}{\vec{t}})$ represents the operator $F$. Moreover, the operator $F$ is unitary since $\|\pi(\vec{w_i})\|_{\C^n}=\|\pi(\vec{w_j})\|_{\C^n}=1$ and $\scal{\pi(\vec{w_i})}{\pi(\vec{w_j})}_{\C^n}=0$.

  \textit{The condition is sufficient:} Suppose that $(\Lam{x}{X}{\vec{t}})$ represents a unitary operator $F:\C^n\to\C^n$. From this we deduce that:
  \[
  (\Lam{x}{X}{\vec{t}})\vec{v_i}\eval\vec{w_i}\
  \]
  For some $\vec{v_i}\in\sem{\basis{X}},\ \vec{w_i}\in\sem{\basis{Y}}$ such that $F(\pi(\vec{v_i})) = \pi(\vec{w_i})$. Then we have:
  \[
    (\Lam{x}{X}{\vec{t}})\vec{v_i}\evalone\vec{t}\ansubst{\vec{v_i}/x}{X} = \vec{t}[\vec{v_i}/x]\eval\vec{w_i}\in\sem{\sharp\basis{Y}},
  \]
  since $\|\vec{w_i}\|=\|F(\pi(\vec{v_i}))\|_{\C^n} = 1$, we can deduce from \Cref{lem:BasesIso}, that $(\Lam{x}{X}{\vec{t}})\in\sem{\sharp\basis{X}\Arr\sharp\basis{Y}}$. Then:
  \[
  \scal{\vec{w_i}}{\vec{w_j}}=\scal{F(\pi(\vec{v_i}))}{F(\pi(\vec{v_j}))}_{\C^n} = 0
  \]\qedhere
\end{proof}

These results can be extended to unitary distributions of lambda abstractions, since $(\Lam{x}{X}{\sum_{i=1}^{n}\alpha_i \vec{t_i}})$ is syntactically different but computationally equivalent to $\sum_{i=1}^{n}\alpha_i (\Lam{x}{X}{\vec{t_i}})$. Ultimately, we generalized one of the main theorems in \cite{DiazcaroGuillermoMiquelValironLICS19}. The inclusion of the basis type in our system allow us to reason more easily about the action of the operators and translate the proof onto a more general case. 
