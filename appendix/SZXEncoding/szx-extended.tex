\section{Semantics of the SZX calculus}%
\label{sec:szx-extended}

We reproduce below the standard interpretation of \szxGND\ diagrams as
density matrices and completely positive
maps~\cite{CJPV19completenessMix,carette_quantum_2021}, modulo scalars.

Let $D_n \subseteq \C^{2^n \times 2^n}$ be the set of n-qubit density
matrices. We define the functor $\interpret{\cdot} : \zxGND \to
\mathbf{CPM(Qubit)}$ which associates to any diagram $\diag : n \to m$ a completely
positive map $\interpret{\diag} : D_n \to D_m$, inductively as
follows.

\[
  \interpret{\diag_1 \otimes \diag_2}
  \;:=\;
  \interpret{\diag_1} \otimes \interpret{D_2}
  %
  \qquad
  %
  \interpret{\diag_2 \circ \diag_1}
  \;:=\;
  \interpret{\diag_2} \circ \interpret{\diag_1}
\]
\[
    \interpret{\tikzfig{szx/elem/hadamard-szx}} :=
      \rho \mapsto V~\rho~V^\dagger
      \text{ where\ }
      V = \sum_{x,y \in \Ftwo^k}
      (-1)^{x \bullet y} \ketbra{y}{x}
\]
\[
    \interpret{\tikzfig{szx/elem/spiderZ-szx}} :=  
      \rho \mapsto V~\rho~V^\dagger
      \text{ where\ }
      V = \sum_{x \in \Ftwo^k}
      e^{i x\bullet\overrightarrow\alpha} \ket{x}^{\otimes m}\bra{x}^{\otimes n}
\]
\[
    \interpret{\tikzfig{szx/elem/spiderX-szx}} :=
    \interpret{\tikzfig{szx/elem/hadamard-szx}}^{\otimes m} \circ
    \interpret{\tikzfig{szx/elem/spiderZ-szx}} \circ
    \interpret{\tikzfig{szx/elem/hadamard-szx}}^{\otimes n}
\]
\[
    \interpret{\tikzfig{szx/elem/ground-szx}} :=
      \rho \mapsto
      \sum_{x \in \Ftwo^k}
      \bra{x} \rho \ket{x}
    %
    \qquad
    %
    \interpret{\tikzfig{szx/elem/ground-szx-rotated}} :=
      \sum_{x \in \Ftwo^k}
      \ketbra{x}{x}
    %
    \qquad
    %
    \interpret{\tikzfig{szx/elem/wire-szx}} :=
    \rho \mapsto \rho
\]
\[
    \interpret{\tikzfig{szx/elem/gather}} :=
      \rho \mapsto \rho
    %
    \qquad
    %
    \interpret{\tikzfig{szx/elem/arrow}} :=  
      \rho \mapsto V~\rho~V^\dagger
      \text{ where\ }
      V = \sum_{x \in \Ftwo^k}
      \ketbra{\sigma(x)}{x}
\]
\[
    \interpret{\tikzfig{szx/elem/cap-szx}} :=
      \rho \mapsto
      \sum_{x \in \Ftwo^k}
      \bra{xx} \rho \ket{xx}
    \qquad
    \interpret{\tikzfig{szx/elem/cup-szx}} :=
      \sum_{x \in \Ftwo^k}
      \ketbra{xx}{xx}
    \qquad
    \interpret{\tikzfig{szx/elem/empty}} := Id_0
\]
\[
    \interpret{\tikzfig{szx/elem/swap-szx}} :=
      \rho \mapsto V~\rho~V^\dagger
      \text{ where\ }
      V = \sum_{x \in \Ftwo^k, y \in \Ftwo^l}
      \ketbra{yx}{xy}
\]
where $\forall u,v\in \mathbb{R}^n, u \bullet v = \sum_{i=1}^m u_i v_i$.

The \szxGND\ calculus defines a set of rewrite rules, shown below.
\[\scalebox{0.9}{\tikzfig{szx/rules/zxRules}}\]
\[
  \tikzfig{szx/rules/numberRule-a}
  \;\overset{\mbox{\scriptsize\GndRemoveRule}}{=}\;
  \tikzfig{szx/elem/empty}
  %
  \qquad
  %
  \tikzfig{szx/rules/HRule-a}
  \;\overset{\mbox{\scriptsize\GndHadamardRule}}{=}\;
  \tikzfig{szx/rules/HRule-b}
  %
  \qquad
  %
  \tikzfig{szx/rules/MergeRule-a}
  \;\overset{\mbox{\scriptsize\GndDiscardRule}}{=}\;
  \tikzfig{szx/rules/MergeRule-b}
  %
  \qquad
  %
  \tikzfig{szx/rules/CNOTRule-a}
  \;\overset{\mbox{\scriptsize\GndCNOTRule}}{=}\;
  \tikzfig{szx/rules/CNOTRule-b}
\]
\[
    \tikzfig{szx/rules/split-gather}
    \ \stackrel{\mathbf{(sg)}}{=}\ %
    \tikzfig{szx/rules/split-gather-2}
    \qquad\qquad
    \tikzfig{szx/rules/gather-split}
    \ \stackrel{\mathbf{(gs)}}{=}\ %
    \tikzfig{szx/rules/gather-split-2}
\]
\[\scalebox{0.9}{\tikzfig{szx/rules/gatherRules}}\]
Additionally, for the arrows restricted to permutations of wires
we have the following rules~\cite{carette_quantum_2021}:
\[\scalebox{0.9}{\tikzfig{szx/rules/permRules}}\]
Finally, since wires with cardinality zero correspond to empty mappings they can be discarded from the diagrams.
\[\scalebox{0.9}{\tikzfig{szx/rules/zeroRules}}\]