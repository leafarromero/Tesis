\section{Application example: QFT}%
\label{sec:qft-example}%

The Quantum Fourier Transform is an algorithm used extensively in quantum computation,
notably as part of Shor's algorithm for integer factorization~\cite{nielsen_chuang_quantum_programming}.
The QFT function operates generically over $n$-qubit states and in general a circuit encoding of it
requires $\bigo(n^2)$ gates.
In this section we present an encoding of the algorithm as a \lambdaD\ term,
followed by the translation into a family of constant-sized diagrams.
The corresponding Proto-Quipper-D program is listed in Appendix~\ref{sec:qft-code}.

The following presentation divides the algorithm into three parts.
The \textit{crot} term applies a controlled rotation over a qubit with a parametrized angle.
\textit{apply\_crot} operates over the last $n-k$ qubits of an $n$-qubit state
by applying a Hadamard gate to the first one and then using it as target
of successive \textit{crot} applications using the rest of the qubits as controls.
Finally, \textit{qft} repeats \textit{apply\_crot} for all values of $k$.
In the terms, we use $n\dots m$ as a shorthand for $\Qrange\ @n\ @m$.
\\

{
   \setlength{\parindent}{0pt}

   \(
      \text{crot}: (n:\nat)\to (\qubit\otimes\qubit)\multimap (\qubit\otimes\qubit)
   \)\\
   \(
     \text{crot}:= \lambda' n^{\nat}.\lambda qs^{\qubit \otimes \qubit}.
       \Qlet{c^\qubit \otimes q^\qubit}{qs}{}
       \Qlet{c^\qubit \otimes q^\qubit}{\mathtt{CNOT}\ c\ (\mathtt{Rz}\ @2^n\ q)}{}
       \mathtt{CNOT}\ c\ (\mathtt{Rz^{-1}}\ @2^n\ q)
   \)\\

   \(
      \text{apply\_crot}: (n:\nat) \to (k:\nat) \to \typeVec{\qubit}{n} \multimap \typeVec{\qubit}{n}
   \)%
   %\vspace*{-1em}
   \begin{flalign*}
      \text{apply\_crot}:=\ & \lambda' n^\nat.\ \lambda' k^\nat.\ \lambda qs^{\typeVec{\qubit}{n}}. & \\
      & \ifz{(n-k)}{qs}{}\\
      & \Qlet{h^{\typeVec{\qubit}{k}}\otimes qs'^{\ \typeVec{\qubit}{n-k}}}
         {\Qsplit\ @k\ @(n-k)\ qs}{}\\
      & \Qlet{q^\qubit \otimes cs^{\typeVec{\qubit}{n-k-1}}}
         {qs'}{}\\
      & \Qlet{fs^{\typeVec{(\qubit\otimes\qubit\multimap\qubit\otimes\qubit)}{(n-k-1)}}}
         {\Qfor{m^\nat}{2..(n-k+1)}{\text{crot }@m}}{\!\!}\\
      & \Qlet{cs'^{\ (\typeVec{\qubit}{n-k-1})}\otimes q'^{\ \qubit}}
         {\Qaccumap\ fs\ (H\ q)\ cs}{}\\
      & \text{concat } h\ (q': cs')
   \end{flalign*}

   \(
      \text{qft}: (n:\nat) \to \typeVec{\qubit}{n}\multimap\typeVec{\qubit}{n}
   \) %
   %\vspace*{-1em}
   \begin{flalign*}
      \text{qft} :=\ & \lambda' n^\nat.\lambda qs^{\typeVec{\qubit}{n}}.
         \Qcompose & \\
         & (\Qfor{k^\nat}{\text{reverse\_vec } @(0..n)}
         {\lambda qs'^{\ \typeVec{\qubit}{n}}.\text{apply\_crot } @n\ @k\ qs'})\ qs
   \end{flalign*}
}

The translation of each term into a family of diagrams is shown below.
We omit the wire connecting the function inputs to the right side
of the graphs for clarity and eliminate superfluous gathers and splitters
using rules $\mathbf{(sg)}$ and $\mathbf{(gs)}$.
Notice that, in contrast to a quantum circuit encoding,
the resulting diagram's size does not depend on the number of qubits $n$.

\[
   \trans{\text{crot}}=\ n \mapsto \tikzfig{examples/qft-crot}
\]
\[
   \trans{\text{apply\_crot}}=\ n,k \mapsto \tikzfig{examples/qft-applycrot}
\]
\[
   \trans{\text{qft}}=\ n \mapsto \tikzfig{examples/qft-main}
\]