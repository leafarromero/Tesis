\section{Encoding programs as diagram families}%
\label{sec:translation}

A key difference between Quipper (and, by extension, Proto-Quipper-D) and $\lambdaD$ is the approach to defining circuits. In Quipper, circuits are an intrinsic part of the language and can be operated upon. In our case, the translation into SZX diagrams will be mediated with a function defined outside the language.

In this section we introduce an encoding of the lambda calculus presented in Section~\ref{sec:fragment}
into families of SZX diagrams with context variables as inputs and term values as outputs.
We split the lambda-terms into those that represent linear mappings between quantum states
and can be encoded as families of SZX diagrams, and parameter terms that can be completely evaluated
at compile-time.

\subsection{Parameter evaluation}

We say a type is \textit{evaluable} if it has the form 
$A = (n_1:\nat) \to \dots \to (n_k:\nat) \to P[n_1,\dots,n_k]$ with $P$ a parameter type.
Since $A$ does not encode a quantum operation,
we interpret it directly into functions over vectors of natural numbers.
The translation of an evaluable type, $\natEval{A}$, is defined recursively as follows:
\[
    \natEval{(n : \nat) \to B[n]} = \N \to \bigcup_{n \in \N} \natEval{B[n]}
    \qquad
    \natEval{\nat} = \N
\]
\[
    \natEval{\typeVec{\nat}{(n : \nat)}} = \N^n
\]

Given a type judgement $\Phi \vdash L : P$ where $P$ is an evaluable type,
we define $\natEval{L}_\Phi$ as the evaluation of the term into a function from parameters into products of natural numbers.
Since the typing is syntax directed, the evaluation is defined directly over the terms as follows:
\[
    \natEval{x}_{x:\nat, \Phi} = x, |\Phi| \mapsto x \quad \natEval{n}_\Phi = |\Phi| \mapsto n \\[7pt]
\]

$
\begin{array}{c}
\natEval{M \square N}_\Phi = |\Phi| \mapsto \natEval{M}_\Phi(|\Phi|) \square \natEval{N}_\Phi(|\Phi|)\\[7pt]
\natEval{M\ ::\ N}_{\Phi} = |\Phi| \mapsto \natEval{M}_\Phi(|\Phi|) \times \natEval{N}_\Phi(|\Phi|)\\[7pt]
\natEval{\vnil^\nat}_{\Phi} = |\Phi| \mapsto [\,]\\
\natEval{\lambda' x^P.M}_\Phi = x,|\Phi| \mapsto \natEval{M}_\Phi(x,|\Phi|)\\[7pt]
\natEval{M @ N}_\Phi = \natEval{M}_\Phi(\natEval{N}_\Phi(|\Phi|), \Phi)\\[7pt]
\natEval{\ifz{L}{M}{N}}_\Phi = |\Phi| \mapsto 
\begin{cases}
    \natEval{M}_\Phi(|\Phi|) \text{ if } \natEval{L}_\Phi(|\Phi|) = 0 \\
    \natEval{N}_\Phi(|\Phi|) \text{ otherwise}
\end{cases}\\[10pt]
\natEval{\Qrange}_\Phi = n,m,|\Phi| \mapsto \bigtimes\limits_{i=n}^{m-1} i \\[10pt]
\natEval{\Qfor{k}{V}{M}}_\Phi = |\Phi| \mapsto \bigtimes\limits_{k\in \natEval{V}_\Phi(|\Phi|)} \natEval{M}_{k:\nat\Phi}(k, |\Phi|)
\end{array}
$


\begin{align*}
    \natEval{\Qlet{x^P :: y^{\typeVec{P}{n}}}{M}{N}}_\Phi &= \\
    |\Phi|\mapsto \natEval{N}_{x:P, y:\typeVec{P}{n}, \Phi}&(y_1,[y_2,\dots,y_n],|\Phi|)\\
    \text{where } [y_1,\dots,y_n] = \natEval{M}_\Phi(|\Phi|)&
\end{align*}


\begin{lemma}%
    \label{lem:eval-type}
    Given an evaluable type $A$ and a type judgement $\Phi \vdash L : A$,
    $\natEval{L}_\Phi \in \bigtimes_{x:P \in \Phi} \natEval{P} \to \natEval{A}$.
\end{lemma}

\begin{lemma}%
    \label{lem:eval-reduction}
    Given an evaluable type $A$,
    a type judgement $\Phi \vdash L : A$, and $M \to N$, then $\natEval{M}_\Phi = \natEval{N}_\Phi$.
\end{lemma}

\subsection{Diagram encoding}

A non-evaluable type has necessarily the form $A = (n_1:\nat) \to \dots \to (n_k:\nat) \to S$,
with $S$ any state type.
We call such types \textit{translatable} since they correspond to terms that
encode quantum operations that can be described as families of diagrams.

We first define a translation $\trans{\cdot}$ from state types
into wire multiplicities as follows.
Notice that due to the symmetries of the SZX diagrams
the linear functions have the same representation as the products.
\[
    \trans\bit = 1
    \qquad
    \trans\qubit = 1
    \qquad
    \trans{\typeVec{A}{(n: \nat)}} = \trans{A}^{\otimes n}
\]
\[
    \trans{A \otimes B} = \trans A \otimes \trans B
    \qquad
    \trans{A \multimap B} = \trans A \otimes \trans B
\]

Given a translatable type judgement
$\Phi, \Gamma \vdash M : (n_1:\nat) \to \dots \to (n_k:\nat) \to S$
we can encode it as a family of SZX diagrams
\[
  n_1, \dots, n_k, |\Phi| \mapsto \tikzfig{judgement-trans}
\].

We will omit the brackets in our diagrams for clarity. In a similar manner to the evaluation, we define the translation $\trans{M}_{\Phi,\Gamma}$ recursively on the terms as follows: 

{\color{red} ARREGLAR ESTA PARTE QUE ESTÁ ILEGIBLE}

\resizebox{.93\linewidth}{!}{\noindent
$
\begin{array}{c c}
    \trans{x}_{\Phi,x:A} = |\Phi|\mapsto \tikzfig{judgements/linear/ax}
    &
    \trans{0}_{\Phi} = |\Phi|\mapsto\tikzfig{judgements/linear/zero}
    \\[10pt]
    \trans{1}_{\Phi} = |\Phi|\mapsto\tikzfig{judgements/linear/one}
    &
    \trans{\meas}_{\Phi} = |\Phi|\mapsto\tikzfig{judgements/linear/meas}
    \\[12pt]
    \trans{\new}_{\Phi} = |\Phi|\mapsto\tikzfig{judgements/linear/new}
    &
    \trans{U}_{\Phi} = |\Phi|\mapsto\tikzfig{judgements/linear/unitary}
    \\[12pt]
    \trans{\star}_\Phi = |\Phi|\mapsto\tikzfig{judgements/linear/skip}
    &
    \trans{\vnil}_\Phi = |\Phi|\mapsto\tikzfig{judgements/linear/skip}
\end{array}
$
}

$
\begin{array}{c}
    \trans{R}_{\Phi} = n,|\Phi|\mapsto\tikzfig{judgements/linear/rotation}
    \\[12pt]
    \trans{M\ @N}_{\Phi,\Gamma} = |\Phi|\mapsto \tikzfig{judgements/linear/applyP}
    \\[11pt]
    \trans{\lambda' x^A . M}_{\Phi,\Gamma} = x,|\Phi| \mapsto \tikzfig{judgements/linear/lambdaP}
    \\[12pt]
    \trans{\lambda x^A . M}_{\Phi,\Gamma} = |\Phi| \mapsto \tikzfig{judgements/linear/lambda}
    \\[12pt]
    \trans{M\ N}_{\Phi,\Gamma,\Delta} = |\Phi|\mapsto \tikzfig{judgements/linear/apply}
    \\[12pt]
    \trans{M; N}_{\Phi, \Gamma, \Delta} = |\Phi|\mapsto \tikzfig{judgements/linear/then}
    \\[14pt]
    \trans{M \otimes N}_{\Phi, \Gamma, \Delta} = |\Phi|\mapsto \tikzfig{judgements/linear/product}    
    \\[14pt]
    \trans{M;_v N}_{\Phi, \Gamma, \Delta} = |\Phi|\mapsto \tikzfig{judgements/linear/then}
    \\[15pt]
    
\end{array}
$

\resizebox{0.93\linewidth}{!}{\noindent
$
\begin{array}{c}
    \trans{let\ x^A \otimes y^B = M\ in\ N}_{\Phi, \Gamma, \Delta} = |\Phi|\mapsto \tikzfig{judgements/linear/let}
    \\[15pt]
    \trans{let\ x^A : y^{\typeVec{A}{n}} = M\ in\ N}_{\Phi, \Gamma, \Delta} = |\Phi|\mapsto \tikzfig{judgements/linear/letVec}
\end{array}
$
}

$
\begin{array}{c}
    \trans{M\!::\!N}_{\Phi, \Gamma, \Delta} = |\Phi|\mapsto \tikzfig{judgements/linear/vector}
    \\[17pt]
    \trans{\Qfor{k}{V}{M}}_{\Phi, \Gamma^n} = |\Phi| \mapsto \tikzfig{judgements/primitives/for}
    \\[17pt]
    \trans{\ifz{L}{M}{N}}_{\Phi, \Gamma} = |\Phi| \mapsto \tikzfig{judgements/linear/ifz}
\end{array}
$


Where $\delta$ is the Kronecker delta and $l = \natEval{L}(|\Phi|)$. Notice that
the $\new$ and $\meas$ operations share the same translation. Although $\new$
can be encoded as a simple wire, we keep the additional node to maintain the
symmetry with the measurement.

The unitary operators $U$ and rotations $R$ correspond to a predefined set of primitives,
and their translation is defined on a by case basis.
The following table shows the encoding of the operators used in this chapter.

{\centering
    \begin{tabular}{|c|c|c|c|c|c|c|}
    \hline
    Name & Rz(n) & Rz$^{-1}$(n) & Rx(n) & Rx$^{-1}$(n) & H & CNOT\\[5pt]
    \hline
    Encoding & 
    $\tikzfig{judgements/unitaries/rz}$ &
    $\tikzfig{judgements/unitaries/rz-inv}$ &
    $\tikzfig{judgements/unitaries/rx}$ &
    $\tikzfig{judgements/unitaries/rx-inv}$ &
    $\tikzfig{judgements/unitaries/hadamard}$ &
    $\tikzfig{judgements/unitaries/cnot}$\\[5pt]
    \hline
    \end{tabular}
    \par
}

The primitives $\Qsplit$, $\Qappend$, $\Qdrop$ and $\Qaccumap$ are translated below. Since vectors are isomorphic to products in the wire encoding, the first three primitives do not perform any operation. We present the translation for these primitives first.

\[
\begin{array}{l}
    \trans{\Qsplit_A}_\Phi = n, m, |\Phi| \mapsto \tikzfig{judgements/primitives/split} \\
    \trans{\Qappend_A}_\Phi = n, m, |\Phi| \mapsto \tikzfig{judgements/primitives/cons} \\
    \trans{\Qdrop}_\Phi = n,|\Phi| \mapsto \tikzfig{judgements/primitives/drop}
\end{array}
\]

For the accumulating map we utilize the construction presented in Lemma~\ref{lem:accumap}, replacing the function box with a function vector input. In the latter we omit the wires and gathers connecting the inputs and outputs of the function to a single wire on the right of the diagram for clarity.

\[
\begin{array}{l}
    \trans{\Qaccumap_{A,B,C}}_\Phi = n, |\Phi| \mapsto \\[5pt]
    \resizebox{0.95\textwidth}{!}{\tikzfig{judgements/primitives/accumap-new}}
\end{array}
\]

Where $\tau_{n,A,B,C}$ is a permutation that rearranges the vectors of functions into tensors of vectors for each parameter and return value. That is, $\tau_{n,A,B,C}$ reorders a sequence of registers $(A,C,B,C)\dots(A,C,B,C)$ into the sequence $(A\dots A)(C\dots C)(B\dots B)(C\dots C)$. It is defined as follows:

\noindent
\(
    \tau_{n,A,B,C}(i) = \begin{cases}
        \imod{i}{k} + a * (\idiv{i}{k})\\
            \qquad\qquad\text{if } \imod{i}{k} < a \\[10pt]
        \imod{i}{k} + c * (\idiv{i}{k}) + a*(n-1)\\
            \qquad\qquad\text{if } a \leq \imod{i}{k} < (a+c) \\[10pt]
        \imod{i}{k} + b * (\idiv{i}{k}) + (a+c)*(n-1)\\
            \qquad\qquad\text{if } (a+c) \leq \imod{i}{k} < (a+c+b)\\[10pt]
        \imod{i}{k} + c * (\idiv{i}{k}) + (a+c+b)*(n-1)\\
            \qquad\qquad\text{if } (a+c+b) \leq \imod{i}{k}\\[10pt]
    \end{cases}
\)

For $i \in [0,(a+c+b+c)*n)$,
where mod and div are the integer modulo and division operators,
$a = \trans{A}$, $b = \trans{B}$, $c = \trans{C}$,
and $k = a+c+b+c$.

As a consequence of Lemma~\ref{lem:list-instantiation-linear},
the number of nodes in the produced diagrams grows linearly
with the size of the input.
Notice that the ZX spiders, the ground, and the Hadamard operator
are only produced in the translations of the quantum primitives.
We may instead have used other variations of the calculus
supporting the scalable extension, such as the ZH calculus~\cite{Backens_2019},
better suited for other sets of quantum operators.

\begin{lemma}%
    \label{lem:trans-reduction}
    The translation procedure is correct in respect to the operational semantics of \lambdaD.
    If $A$ is a translatable type, $\Phi, \Gamma \vdash M : A$, and $M \to N$,
    then $\trans{M}_{\Phi,\Gamma} = \trans{N}_{\Phi,\Gamma}$.
\end{lemma}