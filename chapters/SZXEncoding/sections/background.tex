\section{Background}%
\label{sec:background}

We describe a quantum state as a system of $n$ qubits corresponding to a vector in the $\C^{2^n}$ Hilbert space. We may partition the set of qubits into multi-qubit \textit{registers} representing logically related subsets. Morally, these registers will be interpreted as vectors in our language. Quantum computations under the QRAM model correspond to compositions of unitary operators between these quantum states. Additionally, the qubits may be initialized on a set state and measured. However, we will not provide a probabilistic reduction to perform the measurements. We will only make the distinction through typing.

High-level programs can be encoded in Quipper~\cite{Green2013quipper}, a Haskell-like programming language for describing quantum computations. In this work we use a formalization of the language called Proto-Quipper-D\cite{fu_tutorial_2020} with support for linear and dependent types. Concrete quantum operations correspond to linear functions between quantum states, generated as a composition of primitive operations that can be described directly as a quantum circuit. Generic circuits may have additional parameters that must fixed at compilation time to produce the corresponding quantum circuit.

%subsection szx
\subsection{The Scalable ZX-calculus}%
\label{sec:szx}

The ZX calculus~\cite{vdw_working_cs_zx} is a formal graphical language
that encodes linear maps between quantum states.
Multiple extensions to the calculus have been proposed.
We first present the base calculus with the grounded-ZX extension,
denoted \zxGND~\cite{ground},
to allow us to encode quantum state measurement operations.
A \zxGND\ diagram is generated by the following primitives,
in addition to parallel and serial composition:
\[
    \tikzfig{szx/elem/spiderZ} : n_1 \to m_1
    \qquad
    \tikzfig{szx/elem/spiderX} : n_1 \to m_1
\]
\[
    \tikzfig{szx/elem/hadamard} : 1_1 \to 1_1
    \qquad
    \tikzfig{szx/elem/ground} : 1_1 \to 0_1
\]
\[
    \tikzfig{szx/elem/wire} : 1_1 \to 1_1
    \qquad
    \tikzfig{szx/elem/cup} : 0_1 \to 2_1
\]
\[
    \tikzfig{szx/elem/cap} : 2_1 \to 0_1
    \qquad
    \tikzfig{szx/elem/swap} : 2_1 \to 2_1
    \qquad
    \tikzfig{szx/elem/empty} : 0_1 \to 0_1
\]
where $n_k$ represents the n-tensor of k-qubit registers,
the green and red nodes are called Z and X spiders,
$\alpha \in [0,2\pi)$ is the phase of the spiders,
and the yellow square is called the Hadamard node.
These primitives allow us to encode any quantum operation,
but they can become impractical when working with multiple qubit registers.

The SZX calculus~\cite{carette_szx-calculus_2019,carette_quantum_2021}
is a \textit{Scalable} extension to the ZX-calculus
that generalizes the primitives to work with arbitrarily sized qubit registers.
This facilitates the representation of diagrams with repeated structure
in a compact manner. 
Carette et al.~\cite{carette_quantum_2021} show that the scalable and grounded extensions
can be directly composed.
We will refer to the resulting $\szxGND$-calculus as SZX for simplicity.
%
Bold wires in a SZX diagram are tagged with a non-negative integer representing
the size of the qubit register they carry, and other generators are marked in bold 
to represent a parallel application over each qubit in the register.
Bold spiders with multiplicity $k$ are tagged with $k$-sized vectors of phases
$\overline\alpha = \alpha_1 :: \dots :: \alpha_k$.
The natural extension of the ZX generators correspond to the following primitives:
\[
    \tikzfig{szx/elem/spiderZ-szx} : n_k \to m_k
    \qquad
    \tikzfig{szx/elem/spiderX-szx} : n_k \to m_k
\]
\[
    \tikzfig{szx/elem/hadamard-szx} : 1_k \to 1_k
    \qquad
    \tikzfig{szx/elem/ground-szx} : 1_k \to 0_0
\]
\[
    \tikzfig{szx/elem/wire-szx} : 1_k \to 1_k
    \qquad
    \tikzfig{szx/elem/cup-szx} : 0_0 \to 2_k
\]
\[
    \tikzfig{szx/elem/cap-szx} : 2_k \to 0_0
    \qquad
    \tikzfig{szx/elem/swap-szx} : 1_k \otimes 1_l \to 1_l \otimes 1_k
    \qquad
    \tikzfig{szx/elem/empty} : 0_k \to 0_k
\]
Wires of multiplicity zero are equivalent to the empty mapping.
We may omit writing the wire multiplicity if it can be deduced by context. 

The extension defines two additional generators; a \textit{split} node to split
registers into multiple wires, and a function arrow to apply arbitrary functions
over a register. In this work we restrict the arrow functions to permutations
$\sigma : [0 \dots k) \to [0 \dots k)$ that rearrange the order of the wires.
Using the split node and the wire primitives can derive the rotated version,
which we call a \textit{gather}.
\[
    \tikzfig{szx/elem/split} : 1_{n+m} \to 1_n \otimes 1_m
\]
\[
    \tikzfig{szx/elem/gather} : 1_n \otimes 1_m \to 1_{n+m}
    \qquad
    \tikzfig{szx/elem/arrow} : 1_k \to 1_k
\]

The rewriting rules of the calculus imply that a SZX diagrams can be considered
as an open graph where only the topology of its nodes and edges matters.
%
In the translation process we will make repeated use of the following
reductions rules to simplify the diagrams:
\[
    \tikzfig{szx/rules/split-gather}
    \ \stackrel{\mathbf{(sg)}}{=}\ %
    \tikzfig{szx/rules/split-gather-2}
\]
\[
    \tikzfig{szx/rules/gather-split}
    \ \stackrel{\mathbf{(gs)}}{=}\ %
    \tikzfig{szx/rules/gather-split-2}
\]
  %
In an analogous manner, we will use a legless gather $\tikzfig{szx/gather0}$ to terminate wires with cardinality zero.
This could be encoded as the zero-multiplicity spider $\tikzfig{szx/spider0}$, which represents the empty mapping. Refer to Appendix~\ref{sec:szx-extended} for a complete definition
of the rewriting rules and the interpretation of the SZX calculus.
%
Cf.~\cite{carette_quantum_2021} for a description of the calculus including the generalized arrow generators.

Carette et al.~\cite{carette_quantum_2021} showed that the SZX calculus
can encode the repetition of a function $f : 1_n \to 1_n$ an arbitrary number of times $k\geq 1$ as follows:
\[
    \tikzfig{szx/repeat}
    \quad=\quad%
    \left(\tikzfig{szx/repeat-2}\right)^k
\]
where $f^k$ corresponds to $k$ parallel applications of $f$.
With a simple modification this construction can be used to encode an accumulating map operation.
\begin{lemma}%
    \label{lem:accumap}
    Let $g : 1_n \otimes 1_s \to 1_{m} \otimes 1_s$ and $k\geq 1$, then
    
    \begin{adjustbox}{width=\textwidth}
    $$
    \tikzfig{szx/accumap}
    \quad=\quad%
    \tikzfig{szx/accumap-2}
    $$
    \end{adjustbox}
\end{lemma}
As an example, given a list $N = [n_1, n_2, n_3]$ and a starting accumulator value $x_0$,
this construction would produce the mapping $([n_1, n_2, n_3], x_0) \mapsto ([m_1, m_2, m_3], x_3)$ where
$(m_i, x_i) = g(n_i, x_{i-1})$ for $i \in [1,3]$.

% ------------------------------------------------------------------------------- %

\subsection{SZX diagram families and list instantiation}

We introduce the definition of a family of SZX diagrams $D: \N^k \to \diag$
as a function from $k$ integer \textit{parameters} to SZX diagrams.
We require the structure of the diagrams to be the same for all elements in the family,
parameters may only alter the wire tags and spider phases.
Partial application is allowed, we write $D(n)$ to fix the first parameter of $D$.

%\begin{example}
%    The following example describes a family of diagrams $D$ that applies z-rotations
%    with angle $\sfrac{\pi}{n}$ on $n+1$ qubits.
%    \[D := n \mapsto \tikzfig{szx/family-example}\]
%\end{example}

Since instantiations of a family share the same structure,
we can compose them in parallel by merging the different values of wire tags and spider phases.
We introduce a shorthand for instantiating a family of diagrams on multiple values
and combining the resulting diagrams in parallel.
This definition is strictly more general than the \textit{thickening endofunctor}
presented by Carette et al.~\cite{carette_quantum_2021},
which replicates a concrete diagram in parallel.
A \textit{list instantiation} of a family of diagrams $D: \N^{k+1} \to \diag$ over
a list $N$ of integers is written as $(D(n), n \in N)$.
This results in a family with one fewer parameter, $(D(n), n \in N): \N^k \to \diag$. 
We graphically depict a list instantiation as a dashed box in a diagram, as follows.
\[\tikzfig{szx/list-instantiation-box} := \tikzfig{szx/list-instantiation}\]

The definition of the list instantiation operator is given recursively
on the construction of $D$ in Figure~\ref{fig:list-instantiation}.
On the diagram wires we use $v(N)$ to denote the wire cardinality $\sum_{n \in N} v(n)$,
$\overrightarrow\alpha(N)$ for the concatenation of phase vectors
$\overrightarrow\alpha(n_1) :: \dots :: \overrightarrow\alpha(n_m)$,
and $\sigma(N)$ for the composition of permutations $\bigotimes_{n \in N} \sigma(n)$.
In general, a permutation arrow $\sigma(N,v,w)$ instantiated in concrete values
can be replaced by a reordering of wires between two gather gates%
using the rewrite rule $\bf{(p)}$.

\begin{figure}[tb]
\begin{mdframed}
    Given $D: \N^{k+1} \to \diag$, $N = [n_1, \dots, n_m] \in \N^m$,
    \[
        ((D_1 \otimes D_2)(n), n \in N) := (D_1(n), n \in N) \otimes (D_2(n), n \in N)
    \]
    \[
        \tikzfig{szx/list-instantiation/wire}
        :=
        \tikzfig{szx/list-instantiation/wire-flat}
    \]
    \[
        ((D_2 \circ D_1)(n), n \in N) := (D_2(n), n \in N) \circ (D_1(n), n \in N) 
        %
    \]
    \[
        %
        \tikzfig{szx/list-instantiation/ground}
        :=
        \tikzfig{szx/list-instantiation/ground-flat}
    \]
    \[
        \tikzfig{szx/list-instantiation/hadam}
        :=
        \tikzfig{szx/list-instantiation/hadam-flat}
        %
    \]
    \[  %
        \tikzfig{szx/list-instantiation/arrow}
        :=
        \tikzfig{szx/list-instantiation/arrow-flat}
    \]
    \[
        \tikzfig{szx/list-instantiation/spider-Z}
        :=
        \tikzfig{szx/list-instantiation/spider-Z-flat}
        %
    \]
    \[
        %
        \tikzfig{szx/list-instantiation/spider-X}
        :=
        \tikzfig{szx/list-instantiation/spider-X-flat}
    \]
    \[
        \tikzfig{szx/list-instantiation/gather}
        :=
        \tikzfig{szx/list-instantiation/gather-flat}
    \]
    Where $\sigma(N, v, w) \in \Ftwo^{v(N)+w(N) \times v(N)+w(N)}$ is the permutation defined as the matrix
    \[
        \sigma(N, v, w) = \begin{pmatrix}\sigma_f^N \vert \sigma_g^N \end{pmatrix}, \quad
        \sigma_f^{N} \in \Ftwo^{v(N)+w(N) \times v(N)},
        \ \sigma_g^{N} \in \Ftwo^{v(N)+w(N) \times w(N)}
    \]
    \[
    \begin{array}{llll}
        \sigma_f^{[\,]} = Id_0
        \quad &
        \sigma_f^{n::N'} =
            \begin{pmatrix}
                Id_{v(n)} & 0 \\
                0 & 0 \\
                0 & \sigma_f^{N'}
            \end{pmatrix}
        \quad &
        \sigma_g^{[\,]} = Id_0
        \quad &
        \sigma_g^{n::N'} =
            \begin{pmatrix}
                0 & 0 \\
                Id_{w(n)} & 0 \\
                0 & \sigma_g^{N'}
            \end{pmatrix}
    \end{array}
    \]
    \caption{Definition of the list instantiation operator.}%
    \label{fig:list-instantiation}
\end{mdframed}
\end{figure}

\begin{lemma}%
    \label{lem:list-instantiation}
    For any diagram family $D$, $n_0 : \N$, $N : \N^k$,
    \[
        \tikzfig{szx/list-instantiation-append}
        \;=\;
        \tikzfig{szx/list-instantiation-append-split}
    \]
\end{lemma}

\begin{lemma}%
    \label{lem:list-init-zero}
    A diagram family initialized with the empty list corresponds to the empty map.
    For any diagram family $D$,
    \[
        \tikzfig{szx/list-instantiation-empty}
        \;=\;
        \tikzfig{szx/list-instantiation-empty-flat}
    \]
\end{lemma}

\begin{lemma}%
    \label{lem:list-instantiation-linear}
    The list instantiation procedure on an $n$-node diagram family adds
    $\bigo(n)$ nodes to the original diagram.
\end{lemma}

%subsection proto-quipper fragment
\section{The \texorpdfstring{$\lambdaD$}{lambda sub D} calculus}%
\label{sec:fragment}

We first define a base language from which to build our translation. In this section we present the calculus $\lambdaD$, as a subset of the strongly normalizing Proto-Quipper-D programs. Terms are inductively defined by:
\begin{align*}
    M, N, L :=\; & x \;|\; C \;|\; \R \;|\; \; \U \;|\; 0 \;|\; 1 \;|\; n \;|\; \meas \;|\; \new \;|
            \lambda x^S. M \;|\; M \ N \;|\; \lambda' x^P. M \;\\
        & M\ @\ N \;|\;\star \;|\; M \otimes N \;|\; \Qlet{x^{S_1} \otimes y^{S_2}}{M}{N} \;|\;M;N \;|\; \\
        & \vnil^A \;|\; M :: N \;|\; \Qlet{x^S :: y^{\typeVec S n}}{M}{N} \\
        & M \square N \;|\; \ifz{L}{M}{N} \;|\; \Qfor{k^P}{M}{N}
\end{align*}

Where $C$ is a set of implicit bounded recursive primitives used for operating
with vectors and iterating functions. $n\in\mathbb{N}$, $\square\in\{+, -,
\times, / , \wedge\}$ and $\ifz{L}{M}{N}$ is the conditional that tests for
zero.

Here $\U$ denotes a set of unitary operations and $\R$ is a phase shift gate
with a parametrized angle. In this article we fix the former to the CNOT and
Hadamard (H) gates, and the latter to the arbitrary rotation gates
$R_{z(\alpha)}$ and $R_{x(\alpha)}$.

For the remaining constants, $0$ and $1$ represent bits, $\new$ is used to
create a qubit, and $\meas$ to measure it. $\star$ is the inhabitant of the
$\unit$ type, and the sequence $M;N$ is used to discard it. Qubits can be
combined via the tensor product $M\otimes N$ with $\Qlet{x^{S_1} \otimes
y^{S_2}}{M}{N}$ as its corresponding destructor.
 
The system supports lists; $\vnil^A$ represents the empty list, $M::N$ the
constructor and $\Qlet{x^S :: y^{\typeVec S n}}{M}{N}$ acts as the destructor. 
Finally, the term $\Qfor{k^P}{M}{N}$ allows iterating over parameter lists.

The typing system is defined in Figure~\ref {fig:linear-fragment-typing}. We
write $|\Phi|$ for the list of variables in a typing context $\Phi$. The type
$\typeVec{A}{n}$ represents a vector of known length $n$ of elements of type $A$.

We differentiate between \textit{state contexts} (Noted with $\Gamma$ and
$\Delta$) and \textit{parameter contexts} (Noted with $\Phi$). For our case of
study, parameter contexts consist only of pairs $x:\nat$ or
$x:\typeVec{\nat}{(n:\nat)}$, since they are the only non-linear types of variables
that we manage. Every other variable falls under the state context. The terms
$\lambda x^S. M$ and $MN$ correspond to the abstraction and application which
will be used for state-typed terms. The analogous constructions for
parameter-typed terms are $\lambda' x^P. M$ and $M@N$.

In this sense we deviate from the original Proto-Quipper-D type system, which
supports a single context decorated with indices. Instead, we use a linear and
non-linear approach similar to the work of Cervesato and
Pfenning\cite{Cervesato1996ALL}.

A key difference between Quipper (and, by extension, Proto-Quipper-D) and
$\lambdaD$ is the approach to defining circuits. In Quipper, circuits are an
intrinsic part of the language and can be operated upon. In our case, the
translation into SZX diagrams will be mediated with a function defined outside
the language.

\begin{figure}[hpt]
\begin{mdframed}

    Types:
\(
    A := S \;|\; P \;|\; (n: \nat) \to A[n]
\)

State types:
\(
    S := \bit \;|\; \qubit \;|\;
        \unit \;|\; S_1 \otimes S_2 \;|\; S_1 \multimap S_2 \;|\; \typeVec{S}{(n: \nat)}
\)

Parameter types:
\(
    P := \nat \;|\; \typeVec{\nat}{(n: \nat)}
\)

State contexts:
\(
    \Gamma,\Delta := \cdot \;|\; x : S, \Gamma
\)

Parameter contexts:
\(
    \Phi := \cdot \;|\; x : P, \Phi
\)

    \[
        \infer[\mathsf{ax}]{\Phi, x:A\vdash x:A}{} \qquad
        \infer[\mathsf{ax_0}]{\Phi \vdash 0:\bit}{} \qquad
        \infer[\mathsf{ax_1}]{\Phi \vdash 1:\bit}{} \qquad
    \]
    \[
        \infer[\mathsf{ax}_n]{\Phi\vdash n:\nat}{n\in\mathbb{N}}\qquad
        \infer[\square]{\Phi\vdash M\square N:\nat}
        {\Phi\vdash M:\nat & \Phi\vdash N:\nat}
    \]
    \[
        \infer[\mathsf{meas}]{\Phi \vdash \meas:\qubit\multimap\bit}{} \qquad
        \infer[\mathsf{new}]{\Phi \vdash \new:\bit\multimap\qubit}{} \qquad
        \infer[\mathsf{ax_\unit}]{\Phi\vdash \star: \unit}{}
    \]
    \[
        \infer[\mathsf{u}]{\Phi\vdash \U:\qubit^{\otimes n}\multimap\qubit^{\otimes n}}
        {}
        \qquad
        \infer[\mathsf{r}]{\Phi\vdash \R:(n:\nat)\to\qubit^{\otimes n}\multimap\qubit^{\otimes n}}
        {}
    \]
    \[
        \infer[\multimap_i]{\Phi,\Gamma\vdash\lambda x.M:A\multimap B}{\Phi,\Gamma,x:A\vdash M:B}
        \qquad
        \infer[\rightarrow_i]{\Phi,\Gamma\vdash\lambda' x.M:(n:\nat)\to B}{\Phi,x:\nat,\Gamma\vdash M:B[x]}
    \]
    \[
        \infer[\multimap_e]{\Phi,\Gamma,\Delta\vdash MN:B}{\Phi,\Gamma\vdash M:A\multimap B & \Phi,\Delta\vdash N:A}
        \qquad
        \infer[\rightarrow_e]{\Phi,\Gamma\vdash M @ N:B[n/N]}{\Phi,\Gamma\vdash M:(n:\nat)\to B & \Phi\vdash N:\nat}
    \]
    \[
        \infer[;]{\Phi,\Gamma,\Delta\vdash M;N :B}{\Phi,\Gamma\vdash M:\unit & \Phi,\Delta\vdash N:B}
        \qquad
        \infer[;_{vec}]{\Phi,\Gamma,\Delta\vdash M;_vN :B}{\Phi,\Gamma\vdash M:\typeVec{A}{0} & \Phi,\Delta\vdash N:B}
    \]
    \[
        \infer[\otimes]{\Phi,\Gamma,\Delta\vdash M\otimes N:A \otimes B}{\Phi,\Gamma\vdash M:A & \Phi,\Delta\vdash N:B}
        \qquad
        \infer[let_\otimes]{\Phi,\Gamma,\Delta\vdash \Qlet{x^A \otimes y^B}{M}{N}:C}{\Phi,\Gamma\vdash M:A\otimes B & \Phi,\Delta,x:A,y:B\vdash N: C}
    \]
    \[
        \infer[\vnil]{\Phi\vdash \vnil^A:\typeVec A 0}{}
        \qquad
        \infer[\typeVec{}{}]{\Phi,\Gamma,\Delta\vdash M\!::\!N\ :\typeVec{A}{(n+1)}}{\Phi,\Gamma\vdash M:A & \Phi,\Delta\vdash N: \typeVec A n}
    \]
    \[
        \infer[let_{vec}]{\Phi,\Gamma,\Delta\vdash \Qlet{x^A : y^{\typeVec A n}}{M}{N}:C}{\Phi,\Gamma\vdash M:\typeVec{A}{(n+1)} & \Phi,\Delta,x:A,y:\typeVec A n\vdash N: C}
    \]
    \[
        \infer[for]{\Phi, \Gamma^n \vdash \Qfor{k}{V}{M} : \typeVec {A[k]} n}{n:\nat & \Phi\vdash V:\typeVec \nat n & k:\nat,\Phi,\Gamma \vdash M:A[k]}
    \]
    \[
        \infer[ifz]{\Phi, \Gamma \vdash \ifz{L}{M}{N} : A}{\Phi \vdash L:\nat & \Phi,\Gamma\vdash M:A & \Phi,\Gamma\vdash N: A}
    \]
    \caption{Type system.}%
    \label{fig:linear-fragment-typing}
\end{mdframed}
\end{figure}

Types are divided into two kinds; parameter and state types. Both of these can
depend on terms of type $\nat$. For the scope of this work, this dependence may
only influence the size of vectors types.

Parameter types represent non-linear variable types which are known at the time
of generation of the concrete quantum operations. In the translation into SZX
diagrams, these variables may dictate the labels of the wires and spiders.
Vectors of $\nat$ terms represent their cartesian product.
%
On the other hand, state types correspond to the quantum operations and states
to be computed. In the translation, these terms inform the shape and composition
of the diagrams. Vectors of state type terms represent their tensor product.

In lieu of unbounded and implicit recursion, we define a series of primitive
functions for performing explicit vector manipulation. These primitives can be
defined in the original language, with the advantage of them being strongly
normalizing. The first four primitives are used to manage state vectors, while
the last one is used for generating parameters. For ease of translation some
terms are decorated with type annotations, however we will omit these for
clarity when the type is apparent.

\begin{center}
$
\begin{array}{l}
    \Phi\vdash \Qaccumap_{A,B,C} : (n:\nat) \to \typeVec A n \;\\
    \multicolumn{1}{r}{\multimap\typeVec{(A \multimap C \multimap B \otimes C)}{n} \multimap C \multimap (\typeVec B n) \otimes C} \\
    \Phi\vdash \Qsplit_A : (n:\nat) \to (m:\nat) \to \typeVec{A}{(n+m)} \\
    \multicolumn{1}{r}{\multimap \typeVec A n \otimes \typeVec A m}\\
    \Phi\vdash \Qappend_A : (n:\nat) \to (m:\nat) \to \typeVec{A}{n} \multimap \typeVec A m \\
    \multicolumn{1}{r}{\qquad\multimap \typeVec{A}{(n+m)}}\\
    \Phi\vdash \Qdrop : (n:\nat) \to \typeVec{\unit}{n} \multimap\; \unit \\
    \Phi\vdash \Qrange : (n:\nat)\to (m:\nat)\to \typeVec{\nat}{(m-n)} \\
\end{array}
$
\end{center}

Since every diagram represents a linear map between qubits there is no
representation equivalent to non-terminating terms, even for weakly normalizing
programs. This is the main reason behind the design choice of the primitives
set.
%
We include the operational semantics of the calculus and primitives in Appendix~
\ref{sec:op-semantics}. The encoding of the primitives as Proto-Quipper-D functions
is shown in Appendix~\ref{sec:primitive-trans}. 

We additionally define the following helpful terms based on the previous
primitives to aid in the manipulation of vectors.
Cf. Appendix~\ref{sec:op-semantics} for their definition as \lambdaD-terms.
\begin{center}
$
\begin{array}{l}
    \Phi\vdash \Qmap_{A,B} : (n:\nat) \to \typeVec A n \multimap\; \typeVec{(A \multimap B)}{n} \multimap \typeVec B n \\
    \Phi\vdash \Qfold_{A,C} : (n:\nat) \to \typeVec A n \multimap\; \typeVec{(A \multimap C \multimap C)}{n} \multimap C \\
    \multicolumn{1}{r}{\multimap C }\\
    \Phi\vdash \Qcompose_{A} : (n:\nat) \to \typeVec{(A \multimap A)}{n} \multimap A \multimap A \\
\end{array}
$
\end{center}

The distinction between primitives that deal with state and parameters highlights the inclusion of the $\mathtt{for}$ as a construction into the language instead of a primitive. Since it acts over both parameter and state types, its function is effectively to bridge the gap between the two of them. This operation closely corresponds to the list instantiation procedure presented in the Section~\ref{sec:szx}.

For example, if we take $ns$ to be a vector of natural numbers, and $xs$ a vector of abstractions $R @k (new 0)$. The term $\Qfor{k}{ns}{xs}$ generates a vector of quantum maps by instantiating the abstractions for each individual parameter in $ns$.
