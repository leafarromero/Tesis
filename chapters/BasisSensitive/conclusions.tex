\section{Conclusion}\label{sec:conclusion}

In this chapter we have explored a quantum-control $\lambda$-calculus equipped
with the additional feature of allowing abstractions to be expressed relative
to arbitrary bases, beyond the canonical one.  

The central mechanism enabling this extension is the decoration of
$\lambda$-ab\-strac\-tions and $\mathsf{let}$-constructors with basis annotations,
together with a modified substitution operation that governs how value
distributions decompose across different bases.  
These additions do not increase the expressive power of the original calculus
on which $\lambdaB$ builds, yet they offer a novel perspective for reasoning
about quantum programs and their behaviour under basis changes.

The reduction system coordinates computation through these extended syntactic
constructs and substitutions.  
A key property is that evaluation commutes with the congruence relation,
ensuring that interpreting a value distribution in a different basis does not
affect the computational result.  
Consequently, it is meaningful to reason about terms modulo basis
congruence.

The benefit of this design becomes clear in the realisability model.
The inclusion of atomic types~$\basis{X}$ enables a direct characterisation of
abstractions representing unitary operators---our main semantic result,
generalising the characterisation from~\cite{DiazcaroGuillermoMiquelValironLICS19}.  
Here, the use of basis types yields a simpler and more transparent proof.  

The second major result is the validity of the typing rules presented in
\ref{tab:TypingRules}.  
By deriving these rules from the realisability interpretation, we ensure their
soundness and obtain a principled foundation for a typed programming language
based on the calculus.

Finally, we have illustrated the expressive advantages of the system through two
canonical examples.  
In the case of \emph{Deutsch's algorithm}, the use of basis-aware typing allows
the result to be treated classically, reflecting the algorithm's determinism.  
In the case of \emph{quantum teleportation}, we demonstrated how the
$\mathsf{case}$ construct can simulate gates controlled by Bell-basis
measurements, effectively capturing the deferred-measurement principle within
the calculus.

% En la parte de conclusiones/trabajo futuro, me gustaría meter otra noción de producto interno que juegue bien con las funciones. Con esta definición una función es ortogonal a su eta expansión. O (\x. t_1 + t_2) _|_ (\x. t_2 + t_1)
