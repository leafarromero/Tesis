\section{Operational semantics}\label{sec:reduction}

The reduction system interprets every vector relative to the basis attached to
its abstraction, allowing a step only when the argument can be decomposed on
that basis.  
\ref{tab:Reduction} defines the elementary relation~$\lraneq$, while terms are
considered modulo the congruence of \ref{tab:Congruence}.  
Hence the effective reduction, written~$\lra$, is defined modulo~$\equiv$:
a step $\vv t\lra\vv r$ abbreviates $\vv t\equiv\vv t'\lraneq\vv r'\equiv\vv r$.

\begin{table}[t]
  \begin{align*}
    \text{If }\vv{t}\ansubst{\vv v/x}{X}\text{ is defined,}\quad
    (\Lam{x}{X}{\vv{t}})\vv{v}
    &\lraneq \vv{t}\ansubst{\vv v/x}{X}\\
    \text{If }\vv{t}\ansubst{\vv v/x}{X\otimes Y}\text{ is defined,}\quad
    \LetP{x}{X}{y}{Y}{\vv v}{\vv{t}}
    &\lraneq \vv{t}\ansubst{\vv{v}/x\otimes y}{X\otimes Y}\\
    \gencase{\sum_{i=1}^{n}\alpha_i \vv{v_i}}{\vv{v_1}}{\vv{v_n}}{\vv{t_1}}{\vv{t_n}}
    &\lraneq \sum_{i=1}^{n}\alpha_i \vv{t_i}
  \end{align*}
  \[
    \begin{array}{c}
      \infer{st\lraneq s\vv r}{t\lraneq \vv r}
      \qquad\qquad
      \infer{tv\lraneq \vv rv}{t\lraneq\vv r}
      \qquad\qquad
      \infer{\alpha t\lraneq \alpha \vv r}{t\lraneq\vv r & \alpha\neq 0}
      \\[5pt]
      \infer{(\sum\limits_{i=1}^{j-1}\vv v_i) + \alpha_j t_j + (\sum\limits_{i=j+1}^{n} \alpha_i \vv t_i) \lraneq (\sum\limits_{i=1}^{j-1}\vv v_i) + \alpha_j r_j + (\sum\limits_{i=j+1}^{n} \alpha_i \vv t_i) }{t_j\lraneq\vv r_j & \text{With } \vv v_i \text{ values and } t_j < t_i \text{ in the lexicographical order}}
      \\[12pt]
      \infer{\LetP{x}{A}{y}{B}{t}{\vv{s}}\lraneq
      \LetP{x}{A}{y}{B}{\vv r}{\vv{s}}}{t\lraneq \vv r} 
      \\[5pt]
      \infer{\gencase{\vv t}{\vv v}{\vv w}{\vv{s_1}}{\vv{s_n}}\lraneq
      \gencase{\vv r}{\vv v}{\vv w}{\vv{s_1}}{\vv{s_n}}}{t\lraneq \vv r}
    \end{array}
  \]
  \caption{Reduction system}
  \label{tab:Reduction}
\end{table}

The side condition~$\alpha\neq0$ avoids vacuous steps such as
$0t\lraneq0r$, preventing spurious nondeterminism. On that note,
the condition on the contextual rule of the sum, imposes a strategy
preserving the determinism of~$\lraneq$.

The main reduction rules are $\beta$-reduction, $\mathsf{let}$ binding, and
$\mathsf{case}$ pattern matching.  
Both $\lambda$ and $\mathsf{let}$ bind variables decorated with an orthonormal
basis, indicating which vectors are treated as classical data.  Linear
combinations of these vectors are handled as quantum data, reducing linearly by
the congruence rules of \ref{tab:Congruence}, so that
$t(\alpha\vv s+\beta\vv r)$ is equivalent to
$\alpha\,t\vv s+\beta\,t\vv r$.

The only exception occurs for higher-order terms: since no orthogonal bases are
defined for abstractions, we introduce a special modality~$\AbsBasis$ reducing
linearly over the basis values of a distribution. For instance,

\[
  (\Lam{x}{\AbsBasis}{\vv t})
  \sum_{i=1}^{n}\alpha_i(\Lam{y}{X}{\vv{s_i}})
  \lra
  \sum_{i=1}^n\alpha_i\vv t[\Lam{y}{X}{\vv{s_i}}/x].
\]

The $\mathsf{case}$ pattern matching controls program flow.  
Each operator keeps track of a set of orthogonal values and tests whether the
argument equals each vector, selecting the matching branch.  
If the argument is a linear combination of several vectors, the result is the
corresponding linear combination of the branches.  For example:
\[
  \case{\ket{-}}{\ket{0}}{\ket{1}}{\vv{t_1}}{\vv{t_2}} \lra
  \tfrac{1}{\sqrt{2}}\,\vv{t_1} - \tfrac{1}{\sqrt{2}}\,\vv{t_2}.
\]

The advantage over a conditional branching is the ability to match against
several vectors simultaneously.  For boolean tuples this makes no difference,
as each component can be treated independently.  However, some orthogonal bases
cannot be expressed as products of smaller ones.  This general
$\mathsf{case}$ allows us to match directly against those vectors.  For
example, using the \emph{Bell basis}\footnote{ The four Bell states are
$\Phi^{\pm}=\sqrthalf(\ket{00}\pm\ket{11})$ and
$\Psi^{\pm}=\sqrthalf(\ket{01}\pm\ket{10})$.
}:
\[
  \mathsf{case}\;\vv{v}\;\mathsf{of}\;\{
  \Phi^+ \mapsto \vv{t_1}\ \mid
  \Phi^- \mapsto \vv{t_2}\ \mid
  \Psi^+ \mapsto \vv{t_3}\ \mid
  \Psi^- \mapsto \vv{t_4}\ \}.
\]

This Bell basis is central in quantum communication. In
\ref{sec:teleportation}, we revisit the quantum teleportation protocol,
which relies heavily on these states.

Defining the system in this way yields a strategy within the
\emph{call-by-value} family, namely a generalisation of the
\emph{call-by-basis} strategy introduced in~\cite{ArrighiDowekLMCS17} and
further analysed in~\cite{AssafDiazcaroPerdrixTassonValironLMCS14}. Whereas
call-by-basis fixes a single computational basis for evaluation, our variant,
which we call \emph{call-by-arbitrary-basis}, allows each abstraction to attach
its own orthonormal basis to its argument. Evaluation remains weak: no
reduction takes place under $\lambda$, pairs, $\mathsf{let}$, or
$\mathsf{case}$ constructors.

The congruence relation on terms gives rise to different redexes. We write
$\eval$ for the reflexive–transitive closure of $\lra$. We can show that the
equivalence relation $\equiv$ commutes with $\eval$; in other words,
equivalence is preserved by reduction modulo~$\equiv$.

\begin{theorem}[Reduction preserves equivalence]\label{thm:confluence}
  Let $\vv{t}$ and $\vv{s}$ be closed term distributions with
  $\vv{t}\equiv\vv{s}$. If $\vv{t}\lraneq\vv{t'}$ and $\vv{s}\lraneq\vv{s'}$,
  then there exist term distributions $\vv{r_1}$ and $\vv{r_2}$ such that
  $\vv{t'}\eval\vv{r_1}$, $\vv{s'}\eval\vv{r_2}$, and
  $\vv{r_1}\equiv\vv{r_2}$.
  Diagrammatically:
  \[
    \begin{tikzcd}[row sep=1pt,baseline=(current bounding box.south)]
      & \vv{t}
        \arrow[ld,decorate,decoration={snake, amplitude=0.8, segment length=6pt}, ->]
        &[-3em] \equiv
        &[-3em] \vv{s}
        \arrow[rd,decorate,decoration={snake, amplitude=0.8, segment length=6pt}, ->]
        &\\
      \vv{t'}\arrow[dr,"*",pos=0.9] & & & &
      \vv{s'}\arrow[ld,"*"',pos=0.9] \\
      & \vv{r_1} & \equiv & \vv{r_2} & 
    \end{tikzcd}
    \tag*{\smash{\raisebox{.6\baselineskip}{\qed}}}
  \]
  \qed
\end{theorem}

\begin{proof}
  {\color{red} TRAER DEL PAPER}
\end{proof}

\begin{remark}\label{rmk:determinism}
  Since the reduction relation $\eval$ is defined modulo~$\equiv$, the result
  above is equivalent to stating that there exists a single distribution
  $\vv{r}$ such that $\vv{t'}\eval\vv{r}$ and $\vv{s'}\eval\vv{r}$.
  Moreover, as the elementary reduction~$\lraneq$ is deterministic, the
  reduction relation~$\lra$ is also deterministic; that is, if
  $\vv{t}\lra\vv{r_1}$ and $\vv{t}\lra\vv{r_2}$, then
  $\vv{r_1}\equiv\vv{r_2}$.
\end{remark}
