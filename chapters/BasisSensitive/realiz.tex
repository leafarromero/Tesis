\section{Realizability model}\label{sec:model}

\subsection{Unitary type semantics}
Given the deterministic machine presented in the previous section
(see~\ref{rmk:determinism}), the next
step towards extracting a typing system is to define the sets of values that
characterise its types. To achieve this, we first need to identify the notion
of a type.

Our aim is to define types exclusively inhabited by values of norm~1. The
vectors we wish to study all belong to the \emph{unit sphere}. We write $\Sph$
for the set $\Sph := \{\vv v \in \vv{\Val} \mid \|\vv v\| = 1\}$, which
corresponds to the mathematical representation of quantum data as unit vectors
in a Hilbert space.

\begin{definition}[Unitary value distribution]
  A value distribution $\vv{v}$ is said to be \emph{unitary} if its norm equals~1,
  that is, if $\vv{v}\in\Sph$.
\end{definition}

\begin{definition}[Unitary type]
  A \emph{unitary type} (or simply a \emph{type}) is a notation~$A$ together
  with a set of unitary value distributions, denoted~$\sem{A}$, called the
  \emph{unitary semantics} of~$A$.
\end{definition}


We now turn to type realizers.  
Since the global phase of a quantum state has no physical significance, terms
that differ only by a phase should share the same type.  
Thus, we assign identical types to $\vv t$ and $e^{i\theta}\vv t$, a principle
that guides the definition of realizers.

\begin{definition}[Type realizer]\label{def:Realizer}
  Given a type~$A$ and a term distribution~$\vv t$, we say that~$\vv t$
  \emph{realizes}~$A$ (written~$\vv t \real A$) if there exists a value
  distribution~$\vv v\in\sem A$ such that
  $\vv{t}\eval e^{i\theta}\vv{v}$ for some $\theta\in\R$,
\end{definition}

\begin{table}[t]
  \[
    A := \basis{X} \mid A\Arr A \mid A\times A \mid \sharp A,
    \qquad\text{where $X$ is any orthonormal basis.}
  \]
  \begin{align*}
    \sem{\basis{X}}&:= X\\
    \sem{A\times B}&:= \bigl\{\,(\vv v, \vv w)\;\bigm|\; \vv v\in\sem{A},~\vv w\in\sem{B}\,\bigr\}\\
    \sem{A\Arr B}&:=
    \bigl\{\,\sum_{i=1}^{n}\alpha_i(\Lam{x}{X}{\vv{t_i}})\in\Sph
      \;\bigm|\;
      \forall\vv{w}\in\sem{A},\,
      (\sum_{i=1}^{n}\alpha_i \vv{t_i})\ansubst{\vv{w}/x}{X}\real B
    \bigr\}\\
    \sem{\sharp A}&:= {(\sem{A}^\bot)}^\bot\quad\text{Where: }A^\bot = \{\vv{v}\in\Sph\,\mid\,\scal{\vv{v}}{a} = 0, \forall a\in A\}
  \end{align*}
  \caption{Type notations and semantics}
  \label{tab:UnitaryTypes}
\end{table}

With the notions of unitary types and their realizers in place, we can now
define a concrete approach for our language. We begin with the type grammar
given in~\ref{tab:UnitaryTypes} and build a simple algebra from the sets of
values we aim to represent.
Before that, we introduce the notion of orthogonal complement, which will be
used in the semantics of the~$\sharp$~type:
\[
  \comp{A} = \{\, \vv{v}\in \Sph \mid \scal{\vv{v}}{a} = 0 \text{ for all } a\in A\,\}.
\]

The types $\basis{X}$ serve as atomic types. Each of them represents a finite
set~$X$ of orthogonal vectors forming an orthonormal basis. For instance, a
boolean type can be represented by a basis of size~2, yet we are not restricted
to a single one, since there are infinitely many bases to choose from.

The type~$A\times B$
corresponds to the cartesian product of~$A$ and~$B$.  However, since the syntax
admits pairs only of basic values, the semantic clause
in~\ref{tab:UnitaryTypes} should be
read as an abbreviation for the following expanded form:
\[
  \sem{A\times B}
  := \bigl\{
       \sum_{i=1}^{n}\sum_{j=1}^{m}\alpha_i\beta_j(v_i,w_j)
       \;\bigm|\;
       \sum_{i=1}^{n}\alpha_i v_i\in\sem{A},~
       \sum_{j=1}^{m}\beta_j w_j\in\sem{B}
     \bigr\}.
\]
We stress this for rigour, but for readability we shall continue using the concise notation of pairs.
The arrow type~$A\Arr B$ consists of distributions of $\lambda$-abstractions
that map elements of~$\sem{A}$ to realizers of~$B$.  
Finally, the type~$\sharp A$ denotes the double orthogonal complement of~$A$
intersected with the unit sphere.
It represents quantum data---linear resources that cannot be erased or
duplicated.  
Intuitively, applying the~$\sharp$ operator to a type~$A$ yields the span of the
original interpretation (intersected with the unit sphere).  This captures the
possible linear combinations of values in the unitary semantics of~$A$, as
stated in the following theorem.

\begin{theorem}\label{thm:SharpCharacterization}
  The interpretation of a type~$\sharp A$ contains precisely the
  norm-$1$ linear combinations of values in~$\sem{A}$:
  \(
    \sem{\sharp A}
    = (\sem{A}^\bot)^\bot
    = \Span(\sem{A}) \cap \Sph
  \).
  \qed
\end{theorem}

\begin{proof}
  Proof by double inclusion.
  \begin{description}
    \item[$\Span(\sem{A})\cap\Sph\subseteq (\sem{A}^\bot)^\bot$:] Let $\vv{v}\in\Span(\sem{A})\cap\Sph$. Then $\vv{v}$ is of the form $\sum_{i=1}^{n}\alpha_i \vv{v_i}$ with $\vv{v_i}\in\sem{A}$. Taking $\vv{w}\in\sem{A}^\bot$, we examine the inner product:
    
    \begin{align*}
    \scal{\vv{v}}{\vv{w}} &= \scal{\sum_{i=1}^{n}\alpha_i \vv{v_i}}{\vv{w}}\\
    &= \sum_{i=1}^{n}\overline{\alpha_i}\scal{\vv{v_i}}{\vv{w}}=0
    \end{align*}

    Then $\vv{v}\in(\sem{A}^\bot)^\bot$.

    \item[$(\sem{A}^\bot)^\bot\subseteq \Span(\sem{A})\cap\Sph$:] Reasoning by contradiction, we assume that there is a $\vv{v}\in(\sem{A}^\bot)^\bot$ such that $v\not\in\Span(\sem{A})\cap\Sph$. Since $\vv{v}\not\in\Span(\sem{A})$, $\vv{v}=\vv{w_1} + \vv{w_2}$ such that $\vv{w_1}\in\Span{\sem{A}}$ and $\vv{w_2}$ is a non-null vector which cannot be written as a linear combination of elements of $\sem{A}$. In other words, $\vv{w_2}\in\sem{A}^\bot$. Taking the inner product:
    \[
    \scal{\vv{v}}{\vv{w_2}} = \scal{\vv{w_1}+\vv{w_2}}{\vv{w_2}} = \|\vv{w_2}\|\neq 0
    \]
    Then $\vv{v}\not\in(\sem{A}^\bot)^\bot$. The contradiction stems from assuming $\vv{v}\not\in\Span{\sem{A}}\cap\Sph$.\qedhere
  \end{description}
\end{proof}

The following theorem shows that, as expected for a span, multiple
applications of the~$\sharp$ operator have no further effect beyond the first
application.

\begin{theorem}\label{thm:IdempotentSharp}
  The~$\sharp$ operator is idempotent; that is,
  $\sem{\sharp A} = \sem{\sharp(\sharp A)}$.
  \qed
\end{theorem}

\begin{proof}
  We want to prove that $(((\comp{\sem{A}})^\bot)^\bot)^\bot = (\comp{\sem{A}})^\bot$. For ease of reading, we will write $\comp[n]{A}$ for $n$ successive applications of the operation $\bot$.

  \begin{description}
    \item[$A\subseteq A^{\bot^2}$:] Let $\vv{v}\in A$. Then, for all $\vv{w}\in\comp{A}$, $\scal{\vv{v}}{\vv{w}} = 0$. Then $\vv{v}\in\comp[2]{A}$. With this we have $A\subseteq\comp[2]{A}$.
    
    \item[$A^{\bot^3}\subseteq \comp{A}$:] Let $\vv{u}\in \comp[3]{A}$. Then, for all $\vv{v}\in\comp[2]{A}$, $\scal{\vv u}{\vv v} = 0$. Since we have shown that $A\subseteq \comp[2]{A}$, we have that for all $\vv{w}\in A$, $\scal{\vv u}{\vv w} = 0$. Then $\vv u\in\comp{A}$. With this we have $\comp[3]{A}\subseteq \comp{A}$.
  \end{description}

  With these two inclusions we have that $\comp{A}=\comp[3]{A}$. So we conclude that: $\sem{\sharp(\sharp A)} = \comp[4]{A} = \comp[2]{A} = \sem{\sharp A}$ \qedhere
\end{proof}

\begin{remark}
  A basis type~$\basis{X}$ may consist of value distributions of pairs and can
  therefore be written as the product type of smaller bases. For example, if
  $X=\{\ket{00},\ket{01},\ket{10},\ket{11}\}$, then $\sem{\basis{X}}=\sem{\basis{\B}\times\basis{\B}}$.
  However, this is not possible for entangled bases. A clear example is the
  Bell basis.
\end{remark}

It remains to verify that our type algebra indeed captures the intended sets of
value distributions. The following theorem shows that every member of a type
interpretation has norm~$1$.

\begin{theorem}\label{prop:UnitaryTypes}
  For every type~$A$, $\sem{A}\subseteq\Sph$.
  \qed
\end{theorem}

\begin{proof}
{\color{red} TRAER DEL PAPER}  
\end{proof}

Defining types as sets of values naturally induces a semantic notion of
subtyping. We say that a type~$A$ is a subtype of a type~$B$
(written~$A\leq B$) when the set of realizers of~$A$ is included in that of~$B$.
If the two sets coincide, we say that $A$ and $B$ are \emph{isomorphic}
(written~$A\cong B$).

\begin{example}
  For every type~$A$, we have $A\leq\sharp A$.
  For the base types $\basis{\B}$ and $\basis{\XB}$, however,
  neither inclusion holds:
    $\basis{\B}\not\leq\basis{\XB}$ and
    $\basis{\XB}\not\leq\basis{\B}$.
  Nevertheless, their linear extensions coincide,
  since $\sharp\basis{\B}\cong\sharp\basis{\XB}$.
\end{example}

Although every type is defined by norm-1 value distributions, not every
norm-1 distribution is contained in the interpretation of a type.
For example, consider the distribution
$\tfrac{1}{\sqrt{2}}(\ket{0} + \ket{00})$.
Another example is a linear combination of abstractions defined over different
bases. For instance, the term
\[
  \tfrac{1}{\sqrt{2}}(\Lam{x}{{\B}}{\pauliX{x}})
  + \tfrac{1}{\sqrt{2}}(\Lam{x}{{\XB}}{x})
\]
is not a member of an arrow type, since the bases decorating each abstraction
do not match. However, it is computationally equivalent to the abstraction
$(\Lam{x}{{\B}}{\ket{+}})$, which in turn belongs to the set
$\sem{\basis{\B}\Arr\basis{\XB}}$.


We denote by~$\Type$ the set of all types and by~$\BasisType$ the set of all
basis types~$\basis{X}$.

\subsection{Characterization of unitary operators}

One of the main results of~\cite{DiazcaroGuillermoMiquelValironLICS19}
is the characterisation of $\C^2\to\C^2$ unitary operators using values in
$\sem{\sharp\basis{\B}\Arr\sharp\basis{\B}}$~\cite[Theorem IV.12]{DiazcaroGuillermoMiquelValironLICS19}.
In this subsection we extend this result. Our goal is to prove that abstractions
of type $\sharp\basis{X}\Arr\sharp\basis{Y}$, where both bases have size~$n$,
represent unitary operators
$\C^n\to\C^n$.

Unitary operators are isomorphisms between Hilbert spaces, as they preserve the
structure of the space. With this in mind, the first step is to show that the
members of $\sem{\sharp\basis{X}\Arr\sharp\basis{Y}}$ map basis vectors from
$\sem{\basis{X}}$ onto orthogonal vectors in $\sem{\sharp\basis{Y}}$. In other
words, these abstractions preserve both norm and orthogonality.

\begin{lemma}\label{lem:BasesIso}
  Let $X$ and $Y$ be orthonormal bases of the same finite
  dimension, and let $\Lam{x}{{X}}{\vv t}$ be a closed $\lambda$-abstraction.
  Then $\Lam{x}{{X}}{\vv t}\in\sem{\sharp\basis{X}\Arr\sharp\basis{Y}}$
  if and only if 
  for all $\vv{v_i},\vv{v_j}\in\sem{\basis{X}}$,
  there exist value distributions
  $\vv{w_i},\vv{w_j}\in\sem{\sharp\basis{Y}}$ such that,
    $\vv{t}[\vv{v_i}/x]\eval\vv{w_i}$
    and
    $\vv{t}[\vv{v_j}/x]\eval\vv{w_j}$,
    with
    $\vv{w_i}\perp\vv{w_j}$ whenever $i\neq j$.
    \qed
\end{lemma}

\begin{proof}
  {\color{red} TRAER DEL PAPER}
\end{proof}

Next, we need to bridge the gap between the values in the calculus with vectors in the space $\C^n$. In order to do this, we introduce a meta-language operation $\pi_n$ which translates value distributions into vectors in $\C^n$. The operation simply writes the value in the canonical basis and takes the corresponding coefficients. 

\begin{definition}
Let $\basis{X}$ be an orthonormal basis of size $n$, then for every $\vv{v}\in\sem{\basis{X}}$:
\[
\vv{v}\equiv \sum_{i=1}^{n}\alpha_i\ket{i}
\]
Where $\ket{i}$ is the $n$-th dimensional product of $\ket{0}$ and $\ket{1}$ with $i$ written in binary and $\sum_{i=1}^{n}|\alpha_i|^2=1$. (For example, $\ket{3}$ with $n=4$ is $\ket{0011}$). We define $\pi_n:\sem{\basis{X}}\to\C^n$ as::
\[
\pi_n(\vv{v}) = (\alpha_1,\dotsb ,\alpha_n)
\]
We will omit the subscript when it can be deduced from the context.
\end{definition}

\begin{definition}
We say a $\lambda$-abstraction $(\Lam{x}{X}{\vv{t}})$ represents an operator $F:\C^n\to\C^n$ when:
\[
(\Lam{x}{X}{\vv{t}})\vv{v} \eval \vv{w} \iff F(\pi_n(\vv{v})) = \pi_n(\vv{w})
\]
\end{definition}

Terms such as $\ket{0}$ and $\ket{1}$ are syntactic objects of the
calculus, not vectors of~$\C^2$. Nevertheless, when discussing the behaviour of
terms on Hilbert spaces, we shall occasionally abuse notation and identify
value distributions representing qubits with their corresponding canonical
basis vectors in~$\C^n$. This identification applies only to those
distributions that denote quantum data, not to general syntactic values such as
$\lambda$-abstractions---in particular, to all elements of
$\sem{\sharp\basis{X}}$ for any orthonormal basis~$X$. This identification can
be made precise as follows.

\begin{definition}
  Let $X$ be an orthonormal basis of size~$n$. For every $\vv{v}\in X$, we can
  write
  \(
    \vv{v}\equiv \sum_{i=1}^{n}\alpha_i\ket{i}
  \),
  where $\ket{i}$ denotes the $n$-qubit tuple of $\ket{0}$ and $\ket{1}$
  corresponding to the binary representation of~$i$, and
  $\sum_{i=1}^{n}|\alpha_i|^2=1$. For example, for $n=4$, $\ket{3}$ is
  $\ket{0011}$. We then define $\pi_n:X\to\C^n$ by
  \(
    \pi_n(\vv{v}) = (\alpha_1,\dotsc,\alpha_n)
  \).
\end{definition}

To lighten notation, we shall henceforth omit~$\pi_n$ and use~$\vv v$
directly.

We now establish a correspondence between $\lambda$-abstractions and operators
on~$\C^n$. Intuitively, an abstraction represents a linear operator when its
operational behaviour coincides with the action of that operator on vectors.
Formally:

\begin{definition}
  A $\lambda$-abstraction $\Lam{x}{{X}}{\vv{t}}$ is said to \emph{represent}
  an operator $F:\C^n\to\C^n$ if
    $(\Lam{x}{{X}}{\vv{t}})\vv{v} \eval \vv{w}$
    if and only if
    $F(\vv{v}) = \vv{w}$.
\end{definition}

This definition, together with \ref{lem:BasesIso}, allows us to build a
characterisation of unitary operators as values in
$\sem{\sharp\basis{X}\Arr\sharp\basis{Y}}$.

\begin{theorem}[Characterisation of Unitary Operators]
  Let $X$ and $Y$ be orthonormal bases of size~$n$.
  A closed $\lambda$-abstraction
  $\Lam{x}{{X}}{\vv{t}}\in\sem{\sharp\basis{X}\Arr\sharp\basis{Y}}$
  if and only if it represents a unitary operator
  $F:\C^n\to\C^n$.
\end{theorem}

\begin{proof}
\textit{Necessity.}
  Suppose that $\Lam{x}{{X}}{\vv{t}}\in\sem{\sharp\basis{X}\Arr\sharp\basis{Y}}$.
  Then, by \ref{lem:BasesIso}, for every
  $\vv{v_i}\in\sem{\basis{X}}$ there exists
  $\vv{w_i}\in\sem{\sharp\basis{Y}}$ such that
  $\vv{t}[\vv{v_i}/x]\eval\vv{w_i}$ and
  $\vv{w_i}\perp\vv{w_j}$ whenever $i\neq j$.
  Let $F:\C^n\to\C^n$ be the operator defined by
  $F(\vv{v_i})=\vv{w_i}$.
  By linearity over~$X$, it follows that
  $\Lam{x}{{X}}{\vv{t}}$ represents~$F$.
  Moreover, $F$ is unitary since
  $\|\vv{w_i}\|_{\C^n}=\|\vv{w_j}\|_{\C^n}=1$ and
  $\scal{\vv{w_i}}{\vv{w_j}}_{\C^n}=0$.

  \smallskip
  \textit{Sufficiency.}
  Conversely, suppose that $\Lam{x}{{X}}{\vv{t}}$ represents a
  unitary operator $F:\C^n\to\C^n$.
  Then, for each $\vv{v_i}\in\sem{\basis{X}}$, there exists
  $\vv{w_i}\in\sem{\basis{Y}}$ such that
  $F(\vv{v_i}) = \vv{w_i}$ and
  $(\Lam{x}{{X}}{\vv{t}})\vv{v_i}\eval\vv{w_i}$.
  Hence,
  \(
    (\Lam{x}{{X}}{\vv{t}})\vv{v_i}
    \lraneq
    \vv{t}\ansubst{\vv{v_i}/x}{X}
    = \vv{t}[\vv{v_i}/x]
    \eval \vv{w_i}
    \in \sem{\sharp\basis{Y}}
  \),
  since $\|\vv{w_i}\|=\|F(\vv{v_i})\|_{\C^n}=1$.
  From \ref{lem:BasesIso}, it follows that
  $\Lam{x}{{X}}{\vv{t}}\in\sem{\sharp\basis{X}\Arr\sharp\basis{Y}}$,
  and moreover
  \(
    \scal{\vv{w_i}}{\vv{w_j}}
    = \scal{F(\vv{v_i})}{F(\vv{v_j})}_{\C^n}
    = 0.
  \)
  \qedhere
\end{proof}

This result naturally extends to unitary distributions of
$\lambda$-ab\-strac\-tions, since a term of the form
$\Lam{x}{{X}}{\sum_{i=1}^{n}\alpha_i \vv{t_i}}$ is syntactically different
but computationally equivalent to $\sum_{i=1}^{n}\alpha_i
\Lam{x}{{X}}{\vv{t_i}}$.  Hence, the characterisation of unitary operators
also applies to superpositions of abstractions sharing the same basis~$X$.