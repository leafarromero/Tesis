\subsection{Typing rules}    
Our focus in this section is to enumerate and prove the validity of various typing rules. The objective being to extract a reasonable set of rules to constitute a type system. We first need to lay the groundwork to properly define what does it mean for a typing rule to be valid.

\begin{definition}
    A context (Denoted by capital Greek letters $\Gamma$, $\Delta$) is a mapping $\Gamma:\Var\to\Type\times\BasisType$ assigning a type and basis to each variable in its domain. We note the mapping $\Gamma(x_i)\mapsto(A_i, \basis{X_i})$ as:
    \[
    \Gamma = {x_1}_{\basis{X_1}}:A_1,\dotsb, {x_n}_{\basis{X_n}}:A_n
    \]
\end{definition}

As usual with typing judgements, the context will keep track of the type of free variables of a term. However, since the substitution operation depends on a basis we also wish to include that information. This is not strictly necessary, since the basis a variable is interpreted should not impact on the type. For example the result of the substitution:

\[
(\Lam{x}{\B}{\Pair{x}{y}})\ansubst{\ket{0}/y}{\basis{\B}} = (\Lam{x}{\B}{\Pair{x}{\ket{0}}})
\]

And the substitution:
\[
(\Lam{x}{\B}{(x, y)})\ansubst{\ket{0}/y}{\basis{\XB}} = \frac{1}{\sqrt{2}} ((\Lam{x}{\B}{\Pair{x}{\ket{+}}}) + (\Lam{x}{\B}{\Pair{x}{\ket{-}}}))
\]

Are not syntactically equivalent, but at the same time, they are equivalent under elimination contexts ({\color{red}EL PRIMER CAPÍTULO HABLA DE ESTO ASÍ QUE LO TOMO POR PRESENTADO}). Therefore, since typing via realizability captures computational behaviour, the types will match. We will however keep basis information on the contexts to later simplify our proofs. With this, we can define which substitutions validate a context.

\begin{definition}
    Given a context $\Gamma$ we call the unitary semantics of $\Gamma$, noted $\sem{\Gamma}$, to the set of substitutions such that:
    \begin{align*}
      \sem{\Gamma} &:= 
      \{\sigma\text{ substitution }~\mid~ \dom{\sigma} = \dom{\Gamma}\text{ and } \forall {x_i} \in\dom{\Gamma},\\
      &\Gamma(x_i) = (A_i, \basis{X_i})\Rightarrow \sigma(x_i)=\ansubst{\vec{v_i}/x_i}{\basis{X_i}} \land \vec{v_i}\in\sem{A_i}\}
    \end{align*}
\end{definition}

In order for the calculus to be correct we need to ensure that qubits are treated linearly. The first step is to identify which variables in the context represent quantum data, those will be the ones associated with a type of the form $\sharp A$. We call the subset of $\Gamma$ composed by these variables, its \emph{strict domain}. 

\begin{definition}
    We define the strict domain of a context $\Gamma$, noted $\sdom{\Gamma}$, as:
    \[
    \sdom{\Gamma} := \{x\in\dom{\Gamma} \mid \sem{\Gamma(x)}=\sem{\sharp(\Gamma(x))}\}
    \]
\end{definition}

Here we make use of the idempotence of $\sharp$ (Proposition \ref{prop:IdempotentSharp}) to define the strict domain. 

In order for a typing judgement $\Gamma\vdash \vec{t}: A$ to be valid, it needs to comply with two conditions. First, every free variable in the term $\vec{t}$ must be in the domain of the context $\Gamma$ and every variable in the strict context $\sdom{\Gamma}$ must appear in the term $\vec{t}$. This ensures there is no erasure of information and every variable is accounted. Linear treatment of quantum data is enforced by the substitution.

Second, every substitution in the unitary semantics of $\Gamma$, when applied to the term $\vec{t}$, must yield a term which reduces to a realizer of type $A$. This condition matches the computational behaviour of the term and context to the type. To put it more precisely: 

\begin{definition}
    We say that a typing judgement $\Gamma\vdash \vec t : A$ is valid when:
    \begin{itemize}
        \item $\sdom{\Gamma}\subseteq\FV{\vec t}\subseteq \dom{\Gamma}$
        \item For all $\sigma\in\sem\Gamma$, $\vec{t}\ansubst{\sigma}{}\real A$
    \end{itemize}
\end{definition}

With this definition in mind, we consider a typing rule to be valid, when starting from judgements we assume to be valid we reach a valid conclusion. In table \ref{tab:TypingRules} we enumerate several of these rules. One important thing to note is that there are infinite valid rules, we limit ourselves to listing a subset which could constitute a reasonable typing system for a typed calculus.

{\color{red} ACÁ AGREGAR LA DEFINICIÓN DE JUICIO DE ORTOGONALIDAD}

\begin{table*}
    \small
    $
    \begin{array}{c}
    \infer[\snam{Axiom}]{\TYP{x_X:A}{x}{A}}{\basis{X}\leq A \vee X=\AbsBasis}\quad
    \infer[\snam{Sub}]{\TYP{\Gamma}{\vec{t}}{A'}}{
        \TYP{\Gamma}{\vec{t}}{A} & \SUB{A}{A'}
    }\\
    \noalign{\medskip}
    \infer[\snam{UnitLam}]{
        \TYP{\Gamma}{\sum_{i=1}^n \alpha_i \Lam{x}{A}{\vec{t_i}}}{A\Arr B}
    }{\TYP{\Gamma,x_A: A}{\sum_{i=1}^{n}\alpha_i\vec{t_i}}{B}
    }\\
    \noalign{\medskip}
    \infer[\snam{App}]{\TYP{\Gamma,\Delta}{\vec{s}\,\vec{t}}{B}}{
        \TYP{\Gamma}{\vec{s}}{A\Arr B} & \TYP{\Delta}{\vec{t}}{A}
    }\ 
    \infer[\snam{GlobalPhase}]{\TYP{\Gamma}{e^{i\theta}\cdot\vec{t}}{A}}
    {\TYP{\Gamma}{\vec{t}}{A}}
    \\
    \noalign{\medskip}
    \infer[\snam{Pair}]{\TYP{\Gamma,\Delta}
        {\Pair{\vec{t}}{\vec{s}}}{A\times B}}{
        \TYP{\Gamma}{\vec{t}}{A}&\TYP{\Delta}{\vec{s}}{B}
    }\quad
    \infer[\snam{Weak}]{\TYP{\Gamma,x_A:B}{\vec{t}}{C}}{
        \TYP{\Gamma}{\vec{t}}{B}& \flat{A} & A\leq B
    }
    \\
    \noalign{\medskip}
    \infer[\snam{LetPair}]{\TYP{\Gamma,\Delta} 
        {\LetP{x}{B_1}{y}{B_2}{\vec{t}}{\vec{s}}}{C}}{
        \TYP{\Gamma}{\vec{t}}{A_1\times A_2}&
        \TYP{\Delta,x_{B_1}:A_1,y_{B_2}:A_2}{\vec{s}}{C}
    }\\
    \noalign{\medskip}
    \infer[\snam{LetTens}]{\TYP{\Gamma,\Delta}
        {\LetP{x}{B_1}{y}{B_2}{\vec{t}}{\vec{s}}}{\sharp C}}{
        \TYP{\Gamma}{\vec{t}}{\sharp(A_1\times A_2)}&
        \TYP{\Delta,x_{B_1}:\sharp A_1,y_{B_2}:\sharp A_2}{\vec{s}}{C}
    }\\
    \noalign{\medskip}
    \infer[\snam{Case}]{\TYP{\Gamma,\Delta}
        {\gencase{\vec{t}}{\vec{v_1}}{\vec{v_n}}{\vec{s_1}}{\vec{s_n}}}{A}}{
        \TYP{\Gamma}{\vec{t}}{\genbasis{\vec{v_i}}{i=1}{n}}&
        \forall i,\ \TYP{\Delta}{\vec{s_i}}{A}
    }\\
    \noalign{\medskip}
    \infer[\snam{UnitCase}]{\TYP{\Gamma,\Delta}
        {\gencase{\vec{t}}{\vec{v_1}}{\vec{v_n}}{\vec{s_1}}{\vec{s_n}}}{\sharp A}}{
        \TYP{\Gamma}{\vec{t}}{\sharp \genbasis{\vec{v_i}}{i=1}{n}}&
        \forall i\neq j,\ \SORTH{\Delta}{\vec{s_i}}{\vec{s_j}}{A}
    }\\
    \noalign{\medskip}
    \infer[\snam{Sum}]
        {\TYP{\Gamma}{\sum_{i=1}^{n} \vec{t_i}}{\sharp A}}
        {\forall i\neq j,\, \SORTH{\Gamma}{\vec t_i}{\vec t_j}{A} &
        \sum_{i=1}^{n}|\alpha_i|^2 = 1}
    \\
    \noalign{\medskip}
    \infer[\snam{Contr}]{\TYP{\Gamma,x_A:A}{\vec{t}\,[y:=x]}{B}}{
        \TYP{\Gamma,x_A:A,y_A:A}{\vec{t}}{B}&\flat{A}
    }\ 
    \infer[\snam{Equiv}]{\TYP{\Gamma}{\vec{s}}{A}}{
        \TYP{\Gamma}{\vec{t}}{A}& \vec t\equiv \vec s
    }\\
    \noalign{\medskip}
    \end{array}
    $

    \parbox{\linewidth}{Where the property $\flat$ is defined as: 
    \[\flat X \iff \forall \vec v, \vec w\in\sem{X}, ~ \vec{v}\neq \vec w \Rightarrow \scal{\vec v}{\vec w} = 0
    \]
    }
    \caption{Some valid typing rules}
    \label{tab:TypingRules}
\end{table*}

The main result of this section, is the proof of validity of each of the rules presented on table \ref{tab:TypingRules}.

\begin{theorem}
    The rules in table \ref{tab:TypingRules} are valid.
\end{theorem}

\begin{proof}
    For each typing rule in Table~\ref{tab:TypingRules}~we have to show the typing judgement is valid starting from the premises:
    \begin{description}
    \item[Axiom] It is clear that $\sdom{x:A}\subseteq\{x\}=\dom{x:A}$. Moreover, given $\sigma\in\sem{x_B:A}$, we have $\sigma=\ansubst{\vec v/x}{B}$ for some $\vec{v}\in\sem{A}$. Therefore, $x\ansubst{\sigma}{}=x\ansubst{\vec v}{B}=\vec{v}\real A$.
    
    \item[Sub] Trivial since $\semr{A}\subseteq\semr{A'}$. 

    \item[UnitLam] If the hypothesis is valid, $\sdom{\Gamma,x_A:A}\subseteq \FV{\sum_{i=1}^{n}\alpha_i \vec t_i}\subseteq \dom{\Gamma,x_A:A}$. It follows that $\sdom{\Gamma}\subseteq \FV{\sum_{i=1}^{n}\alpha_i \Lam{x}{A}{\vec t_i}}\subseteq \dom{\Gamma}$. Given $\sigma\in\sem{\Gamma}$, we want to show that $(\sum_{i=1}^{n}\alpha_i \Lam{x}{A}{\vec t_i})\ansubst{\sigma}{}\real A\Arr B$. Let $\vec v\in\sem{A}$, then:
    
    \begin{align*}
        (\sum_{i=1}^{n} (\Lam{x}{A}{\vec t_i}))\ansubst{\sigma}{} \vec v&= (\sum_{j=1}^{m} \beta_j (\sum_{i=1}^{n} \alpha_i (\Lam{x}{A}{\vec t_i}) [\sigma_i])) \vec{v} \\
        &= (\sum_{i=1}^{n}\sum_{j=1}^{m}\alpha_i\beta_j (\Lam{x}{A}{\vec t_i[\sigma_j]}))\vec v\\
        &\to \sum_{i=1}^{n}\sum_{j=1}^{m}\alpha_i\beta_j \vec{t_i}[\sigma_j]\ansubst{\vec v/x}{A}\\
        &=\sum_{i=1}^{n}\alpha_i \vec{t_i}\ansubst{\sigma}{}\ansubst{\vec v/x}{A}\\
        &=(\sum_{i=1}^{n}\alpha_i \vec{t_i})\ansubst{\sigma}{}\ansubst{\vec v/x}{A}\qquad{\text{By lemma \ref{lem:distributiveSubstitution}}}
    \end{align*}
    
    Considering that $\ansubst{\sigma}{}\in\sem{\Gamma}$, then we have that $\ansubst{\sigma}{}\ansubst{\vec v/x}{A}\in\sem{\Gamma,x_A:A}$. Since we assume $\TYP{\Gamma, x_A:A}{\sum_{i=1}^{n}\alpha_i\vec t_i}{B}$, then $\vec{t_i}\ansubst{\sigma}{}\ansubst{\vec v/x}{A}\real B$. Finally, we can conclude that the distribution: $\sum_{i=1}^{n}\alpha_i \Lam{x}{A}{\vec t_i}\in\sem{A\Arr B}$.

    \item[App] If the hypotheses are valid, then:
    \begin{itemize}
        \item $\sdom{\Gamma}\subseteq \FV{\vec s}\subseteq \dom{\Gamma}$ and $\vec s \ansubst{\sigma_\Gamma}{}\Vdash A\Arr B\ \forall \sigma_\Gamma\in\sem{\Gamma}$.
        \item $\sdom{\Delta}\subseteq \FV{\vec t}\subseteq \dom{\Delta}$ and $\vec t\ansubst{\sigma_\Delta}{}\Vdash A\ \forall\sigma_\Delta\in\sem{\Delta}$.
    \end{itemize}
    
    From this, we can conclude that $\sdom{\Gamma,\Delta}\subseteq \FV{\vec s \vec t}\subseteq \dom{\Gamma,\Delta}$. Given $\sigma\in\sem{\Gamma,\Delta}$, we can observe that $\sigma=\sigma_\Gamma,\sigma_\Delta$ for some $\sigma_\Gamma\in\sem{\Gamma}$ and $\sigma_\Delta\in\sem{\Delta}$. Then we have:
    
    \begin{align*}
        (\vec{t}\vec{s})\ansubst{\sigma}{} &= (\vec{t}\vec{s})\ansubst{\sigma_\Gamma}{}\ansubst{\sigma_\Delta}{}\\
        &=(\sum_{i=i}^{n}\alpha_i (\vec{t}\vec{s})[\sigma_{\Gamma i}])\ansubst{\sigma_\Delta}{}\\
        &=\sum_{j=1}^{m} \beta_j (\sum_{i=1}^{n} \alpha_i (\vec{t} \vec{s})[\sigma_{\Gamma i}])[\sigma_{\Delta j}]\\
        &=\sum_{i=1}^{n}\sum_{j=1}^{m} \alpha_i \beta_j \vec{t}\,[\sigma_{\Gamma i}][\sigma_{\Delta j}] \vec{s}\,[\sigma_{\Gamma i}][\sigma_{\Delta j}]\\
        &=\sum_{i=1}^{n}\sum_{j=1}^{m} \alpha_i \beta_j \vec{t}\,[\sigma_{\Gamma i}]\vec{s}\,[\sigma_{\Delta j}]\\
        &\equiv (\sum_{i=1}^{n}\alpha_i\vec{t}[\sigma_{\Gamma i}])(\sum_{j=1}^{m} \beta_j \vec{s}[\sigma_{\Delta j}])\\
        &=\vec{t}\ansubst{\sigma_\Gamma}{} \vec{s}\ansubst{\sigma_\Delta}{}\\
        &\eval (e^{i\theta_{1}} \vec{v}) (e^{i\theta_{2}} \vec{w})\qquad\text{Where: } \vec{v}\in\sem{A\Arr B}, \vec{w}\in\sem{A}\\
        &\equiv e^{i\theta} (\vec{v} \vec{w})\qquad\text{With: }\theta=\theta_1 + \theta_2\\
        &\evalone e^{i\theta}\vec r\qquad\text{Where: } \vec{r}\real B
    \end{align*}
    
    Then we can conclude that $(\vec{t}\vec{s})\ansubst{\sigma}{}\real B$.
    
    \item[Pair] If the hypotheses are valid, then:

    \begin{itemize}
        \item $\sdom{\Gamma}\subseteq \FV{\vec s}\subseteq \dom{\Gamma}$ and $\vec s \ansubst{\sigma_\Gamma}{}\Vdash A\ \forall \sigma_\Gamma\in\sem{\Gamma}$.
        \item $\sdom{\Delta}\subseteq \FV{\vec t}\subseteq \dom{\Delta}$ and $\vec t\ansubst{\sigma_\Delta}{}\Vdash B\ \forall \sigma_\Delta\in\sem{\Delta}$.
    \end{itemize}
    
    From this, we can conclude that $\sdom{\Gamma,\Delta}\subseteq \FV{(\vec s, \vec t)}\subseteq \dom{\Gamma,\Delta}$. Given $\sigma\in\sem{\Gamma,\Delta}$, we can observe that $\sigma=\sigma_\Gamma,\sigma_\Delta$ for some  $\sigma_\Gamma\in\sem{\Gamma}$ and $\sigma_\Delta\in\sem{\Delta}$. Then we have:

    \begin{align*}
        \Pair{\vec{t}}{\vec{s}}\ansubst{\sigma}{} &= \Pair{\vec{t}}{\vec{s}}\ansubst{\sigma_\Gamma}{}\ansubst{\sigma_\Delta}{}\\
        &=\sum_{j=1}^{m} \beta_j (\sum_{i=1}^{n} \alpha_i \Pair{\vec{t}}{\vec{s}}[\sigma_{\Gamma i}])[\sigma_{\Delta j}]\\
        &\equiv\sum_{i=1}^{n}\sum_{j=1}^{m} \alpha_i \beta_j \Pair{\vec{t}\,[\sigma_{\Gamma i}][\sigma_{\Delta j}]}{\vec{s}\,[\sigma_{\Gamma i}][\sigma_{\Delta j}]}\\
        &=\sum_{i=1}^{n}\sum_{j=1}^{m} \alpha_i \beta_j \Pair{\vec{t}\,[\sigma_{\Gamma i}]}{\vec{s}\,[\sigma_{\Delta j}]}\\
        &=\Pair{\sum_{i=1}^{n} \alpha_i \vec{t}\, [\sigma_{\Gamma i}]}{\sum_{j=1}^{m} \beta_j \vec{s}\, [\sigma_{\Delta j}]}\\
        &=\Pair{\vec{t}\ansubst{\sigma_\Gamma}{}}{\vec{s}\ansubst{\sigma_\Delta}{}}\\
        &\eval \Pair{e^{i\theta_1}\cdot\vec v}{e^{i\theta_2}\cdot\vec w}\qquad\text{Where: }\vec{v}\in\sem{A}, \vec{w}\in\sem{B}\\
        &= e^{i\theta} \Pair{\vec{v}}{\vec{w}}\qquad\text{Where: }\vec{v}\in\sem{A},\vec{w}\in\sem{B}
    \end{align*}
    
    From this we can conclude that $\Pair{\vec t}{\vec{s}}\ansubst{\sigma}{}\real A\times B$. Finally, $\TYP{\Gamma,\Delta}{\Pair{\vec{t}}{\vec{s}}}{A\times B}$
    
    \item[LetPair] If the hypotheses are valid, then:
    \begin{itemize}
        \item $\sdom{\Gamma}\subseteq \FV{\vec t} \subseteq \dom{\Gamma}$ and $\vec t \ansubst{\sigma_\Gamma}{}\Vdash A\times B\ \forall \sigma_\Gamma\in\sem\Gamma$
        \item $\sdom{\Delta, {x_1}_{B_1}:A_1, {x_2}_{B_2}:A_2}\subseteq\FV{\vec s}$
        \item $\FV{\vec{s}}\subseteq \dom{\Delta,{x_1}_{B_1}:A_1, {x_2}_{B_2}:A_2}$
        \item $\vec s \ansubst{\sigma_\Delta}{}\Vdash C\ \forall \sigma_\Delta\in\sem{\Delta, {x_1}_{B_1}:A_1, {x_2}_{B_2}:A_2}$
    \end{itemize}
    From this, we can conclude that:
    \begin{itemize}
        \item $\sdom{\Gamma,\Delta}\subseteq\FV{\LetP{x}{B_1}{y}{B_2}{\vec{s}}{\vec{t}}}$
        \item $\FV{\LetP{x}{B_1}{y}{B_2}{\vec{s}}{\vec{t}}}\subseteq\dom{\Gamma,\Delta}$
    \end{itemize}
    
    Given $\sigma\in\sem{\Gamma,\Delta}$, we have that $\ansubst{\sigma}{}=\ansubst{\sigma_\Gamma}{},\ansubst{\sigma_\Delta}{}$ for some $\sigma_\Gamma\in\sem\Gamma$ and $\sigma_\Delta\in\sem\Delta$. Then we have:
    \begin{align*}
        (&\LetP{x}{B_1}{y}{B_2}{\vec{t}}{\vec{s}})\ansubst{\sigma}{} = \\
        &(\LetP{x}{B_1}{y}{B_2}{\vec{t}}{\vec{s}})\ansubst{\sigma_\Gamma}{}\ansubst{\sigma_\Delta}{}\\
        &= \sum_{i=1}^{n}\sum_{j=1}^{m}\alpha_i\beta_j(\LetP{x}{B_1}{y}{B_2}{\vec{t}}{\vec{s}})[\sigma_{\Gamma i}][\sigma_{\Delta j}]\\
        &\equiv \LetP{x}{B_1}{y}{B_2}{\sum_{i=1}^{n}\alpha_i[\sigma_{\Gamma i}]\vec{t}}{\sum_{j=1}^{m}\beta_j \vec{s}[\sigma_{\Delta j}]}\\
        &= \LetP{x}{B_1}{y}{B_2}{\vec{t}\ansubst{\sigma_\Gamma}{}}{\vec{s}\ansubst{\sigma_\Delta}{}}\\
        &\eval \LetP{x}{B_1}{y}{B_2}{e^{i\theta}\cdot\Pair{\vec{v}}{\vec{w}}}{\vec{s}\ansubst{\sigma_\Delta}{}}\\
        &\hspace*{4cm}{\text{Where: }}\vec{v}\in\sem{A},\vec{w}\in\sem{B}\\
        &\evalone e^{i\theta_1}\cdot(\vec{s}\ansubst{\sigma_\Delta}{}\ansubst{\Pair{\vec{v}}{\vec{w}}/x_1\otimes x_2}{B_1\otimes B_2})\\
        &= e^{i\theta_1}\cdot(\vec{s}\ansubst{\sigma_\Delta}{}\ansubst{\vec{v}/x_1}{B_1}\ansubst{\vec{w}/x_2}{B_2})\\
        &\eval e^{i\theta_1}\cdot (e^{i\theta_2}\cdot \vec{u})\qquad\text{Where: }\vec{u}\in\sem{C}\\
        &\equiv e^{i\theta}\cdot \vec{u}\qquad\text{Where: }\theta=\theta_1 + \theta_2
    \end{align*}
    
    Since $\ansubst{\sigma_\Delta}{}\ansubst{\vec{v}/x_1}{B_1}\ansubst{\vec{w}/x_2}{B_2}\in\sem{\Delta,{x_1}_{B_1}:A_1,{x_2}_{B_2}:A_2}$, then we can conclude that $(\LetP{x}{B_1}{y}{B_2}{\vec{t}}{\vec{s}})\ansubst{\sigma}{}\real C$.

    \item[LetTens] If the hypotheses are valid then:
    \begin{itemize}
        \item $\sdom{\Gamma}\subseteq \FV{\vec t} \subseteq \dom{\Gamma}$ and $\vec t \ansubst{\sigma}{}\Vdash A\otimes B\ \forall \sigma\in\sem\Gamma$
        \item $\sdom{\Delta, {x_1}_{B_1}:\sharp A_1, {x_2}_{B_2}:\sharp A_2}\subseteq \FV{\vec s}$
        \item $\subseteq \dom{\Delta,{x_1}:{B_1}, {x_2}_{B_2}:A_2}$
        \item $\vec s \ansubst{\sigma}{}\Vdash \sharp C\ \forall \sigma\in\sem{\Delta, {x_1}_{B_1}:\sharp A_1, {x_2}_{B_2}:\sharp A_2}$
    \end{itemize}
    
    From this we can conclude that:
    \begin{itemize}
        \item $\sdom{\Gamma,\Delta}\subseteq\FV{\LetP{x_1}{B_1}{x_2}{B_2}{\vec{t}}{\vec{s}}}$
        \item $\FV{\LetP{x_1}{B_1}{x_2}{B_2}{\vec{t}}{\vec{s}}}\subseteq\dom{\Gamma,\Delta}$
    \end{itemize}
    
    Given $\sigma\in\sem{\Gamma,\Delta}$, we have that $\ansubst{\sigma}{}=\ansubst{\sigma_\Gamma}{},\ansubst{\sigma_\Delta}{}$ for some $\sigma_\Gamma\in\sem\Gamma$ and $\sigma_\Delta\in\sem\Delta$. Using the first hypothesis we have that, $\vec t\ansubst{\sigma_\Gamma}{}\real \sharp(A_1\times A_2)$, from \ref{prop:SharpCharacterization} we have that:
    
    \[\vec t\ansubst{\sigma_\Gamma}{}\eval e^{i\theta_1}\cdot\vec{u}=e^{i\theta_1}\cdot(\sum_{k=1}^{l} \gamma_k \Pair{\vec v_k}{\vec u_k})\] 
    
    With:
    \begin{itemize}
        \item $\sum_{k=1}^{l} |\gamma_k|^2 = 1$
        \item $\forall k,\ \vec v_k\in\sem{A_1},\ \vec u_k\in\sem{A_2}$
        \item $\forall k\neq l, \scal{\Pair{\vec{v_k}}{\vec{u_k}}}{\Pair{\vec{v_l}}{\vec{u_l}}}= 0$
    \end{itemize}
    
    Then:
    \begin{align*}
        (&\LetP{x_1}{B_1}{x_2}{B_2}{\vec{t}}{\vec{s}})\ansubst{\sigma}{} \\
        &= \LetP{x_1}{B_1}{x_2}{B_2}{\vec{t}}{\vec{s}}\ansubst{\sigma_\Gamma}{}\ansubst{\sigma_\Delta}{}\\
        &=\sum_{i=1}^{n}\sum_{j=1}^{m}\LetP{x_1}{B_1}{x_2}{B_2}{\vec{t}}{\vec{s}}\ [\sigma_{\Gamma i}][\sigma_{\Delta j}]\\
        &\equiv \LetP{x_1}{B_1}{x_2}{B_2}{\sum_{i=1}^{n}\alpha_i\vec{t}\ [\sigma_{\Gamma i}]}{\sum_{j=1}^{m}\beta_j \vec{s}\ [\sigma_{\Delta j}]}\\
        &=\LetP{x_1}{B_1}{x_2}{B_2}{\vec{t}\ansubst{\sigma_\Gamma}{}}{\vec{s}\ansubst{\sigma_\Delta}{}}\\
        &\eval\LetP{x_1}{B_1}{x_2}{B_2}{e^{i\theta_1}\cdot\vec{u}}{\vec{s}\ansubst{\sigma_\Delta}{}}\\
        &\evalone e^{i\theta_1}\cdot(\vec{s}\ansubst{\sigma_\Delta}{}\ansubst{\vec{u}/x_1\otimes x_2}{B_1\otimes B_2})\\
        &=e^{i\theta_1}\cdot(\sum_{k=1}^{l}\gamma_k\vec{s}\ansubst{\sigma_\Delta}{}\ansubst{\vec{v_k}/x}{B_1}\ansubst{\vec{u_k}/y}{B_2})\\
        &\eval e^{i\theta_1}\cdot(\sum_{k=1}^{l}\gamma_k e^{i\rho_k} \vec{w_k})\\
    \end{align*}

    Since $\vec{s}\ansubst{\sigma_\Delta}{}\ansubst{\vec{v_k}/x}{B_1}\ansubst{\vec{u_k}/y}{B_2}\in\sem{\Delta, x_{B_1}:\sharp A_1, y_{B2}:\sharp A_2}$, for every $k$, then $\vec{w_k}\in\sem{C}$. It remains to be seen that the term has norm-$1$, $\|\sum_{k=1}^{l}\gamma_k e^{i\rho_k} \vec{w_k}\|=1$. For that, we observe:
    \begin{align*}
        \|&\sum_{k=1}^{l}\gamma_k e^{i\rho_k} \vec{w_k}\| \\
        &= \scal{\sum_{k=1}^{l}\alpha_i e^{i\rho_k} \vec{w_k}}{\sum_{k'=1}^{l}\gamma_{k'} e^{i\rho_{k'}} \vec{w_{k'}}}\\
        &= \sum_{k=1}^{l}\sum_{k'}^{l}\overline{\gamma_k e^{i\rho_k}}\  \gamma_{k'} e^{i\rho_{k'}}\scal{\vec{w_k}}{\vec{w_{k'}}}\\
        &=\sum_{k=1}^{l}\sum_{k'=1}^{l}\overline{\gamma_k e^{i\rho_k}}\ \gamma_{k'} e^{i\rho_{k'}} \scal{\vec{v_k}}{\vec{v_{k'}}}\scal{\vec{u_k}}{\vec{u_{k'}}}\quad(\text{from Lemma \ref{lem:UnitPreserTens}})\\
        &= \sum_{k=1}^{k}\sum_{k'=1}^{l}\overline{\gamma_k e^{i\rho_k}}\  \gamma_{k'} e^{i\rho_{k'}} \scal{\Pair{\vec{u_k}}{\vec{v_k}}}{\Pair{\vec{u_{k'}}}{\vec{v_{k'}}}}\quad(\text{from Prop \ref{prop:InnerProdPairs}})\\
        &=\sum_{k=1}^n \overline{\gamma_k e^{i\rho_k}}\ \gamma_k e^{i\rho_k} \scal{\Pair{\vec{v_k}}{\vec{u_k}}}{\Pair{\vec{v_k}}{\vec{u_k}}} \\
        & \quad + \sum_{k,k'=1; k\neq k'}^n \overline{\gamma_k e^{i\rho_k}}\  \gamma_{k'} e^{i\rho_{k'}} \scal{\Pair{\vec{v_k}}{\vec{u_k}}}{\Pair{\vec{v_{k'}}}{\vec{u_{k'}}}}\\
        &= \sum_{k=1}^n \overline{\gamma_k e^{i\rho_k}}\ \gamma_k e^{i\rho_k} + 0 \\
        &= \sum_{k=1}^{l} |\gamma_k|^2 |e^{i\rho_k}|^2 = 1
    \end{align*}

    Then $\sum_{i=1}^{n}\alpha_i\vec{w_i}\in\sem{\sharp C}$. Finally, we can conclude that: 
    \[(\LetP{x_1}{B_1}{x_2}{B_2}{\vec{t}}{\vec{s}})\ansubst{\sigma}{}\real{\sharp C}\]

    \item[Case] If the hypotheses are valid then:
    \begin{itemize}
        \item $\sdom{\Gamma}\subseteq \FV{\vec{t}}\subseteq \dom{\Gamma}$
        \item For every $\sigma_\Gamma\in\sem{\Gamma}$, $\vec{t}\ansubst{\sigma_\Gamma}{}\real\genbasis{\vec{v_i}}{i=1}{n}$
        \item For every $i\in\{0,\dotsb ,n\}, \sdom{\Delta}\subseteq \FV{\vec{s_i}}\subseteq \dom{\Delta}$
        \item For every $i\in\{0,\dotsb ,n\}, \sigma_\Delta\in\sem{\Delta}$, $\vec{s_i}\ansubst{\sigma_\Delta}{}\real A$
    \end{itemize}

    From this we can conclude that:
    
    \begin{itemize}
        \item $\sdom{\Gamma,\Delta}\subseteq \FV{\gencase{\vec{t}}{\vec{v_1}}{\vec {v_n}}{\vec{s_1}}{\vec{s_n}}}$
        \item $\FV{\gencase{\vec{t}}{\vec{v_1}}{\vec {v_n}}{\vec{s_1}}{\vec{s_n}}}\subseteq \dom{\Gamma,\Delta}$
    \end{itemize}


    
    Then, given $\sigma\in\sem{\Gamma,\Delta}$, we have that $\ansubst{\sigma}{}=\ansubst{\sigma_\Gamma}{}\ansubst{\sigma_\Delta}{}$ for some $\sigma_\Gamma\in\sem{\Gamma}$ and $\sigma_\Delta\in\sem{\Delta}$. Using the first hypothesis we have that, $\vec{t}\ansubst{\sigma_\Gamma}{}\eval e^{i\theta_1}\cdot\vec{v_k}$ for some $k\in\{1,\dotsb ,n\}$. From the second hypothesis we have that $\vec{s_i}\ansubst{\sigma_\Delta}{}\eval e^{i\rho_i}\cdot\vec{u_i}\in\sem{A}$ for $i\in\{1,\dotsb , n\}$. Therefore:

    \begin{align*}
        (&\gencase{\vec{t}}{\vec v_1}{\vec v_n}{\vec{s_1}}{\vec{s_n}})\ansubst{\sigma}{}\\ 
        &= (\gencase{\vec{t}}{\vec v_1}{\vec v_n}{\vec{s_1}}{\vec{s_n}})\ansubst{\sigma_\Gamma}{}\ansubst{\sigma_\Delta}{}\\
        &= (\sum_{i=1}^{n}\alpha_i \gencase{\vec{t}[\sigma_{\Gamma i}]}{\vec v_1}{\vec v_n}{\vec{s_1}}{\vec{s_n}})\ansubst{\sigma_\Delta}{} \\
        &\equiv (\gencase{\sum_{i=1}^{n} \alpha_i \vec{t}[\sigma_{\Gamma i}]}{\vec v_1}{\vec v_n}{\vec{s_1}}{\vec{s_n}})\ansubst{\sigma_\Delta}{}\\
        &=(\gencase{\vec{t}\ansubst{\sigma_\Gamma}{}}{\vec{v_1}}{\vec{v_n}}{\vec{s_1}}{\vec{s_n}})\ansubst{\sigma_\Delta}{}\\
        &\eval(\gencase{e^{i\theta_1}\cdot\vec{v_k}}{\vec{v_1}}{\vec{v_n}}{\vec{s_1}}{\vec{s_n}})\ansubst{\sigma_\Delta}{}\\
        &\evalone e^{i\theta_1}\cdot(\vec{s_k}\ansubst{\sigma_\Delta}{})\\
        &\eval e^{i\theta_1}\cdot(e^{i\rho_k}\cdot\vec{u_k})\qquad\text{Where: }\vec{u_k}\in\sem{A}\\
        &\equiv e^{i\theta}\cdot\vec{u_k}\qquad\text{With: }\theta=\theta_1 +\theta_2
    \end{align*}
    
    Since we pose no restriction on $k$, we can conclude that:
    \[(\gencase{\vec{t}}{\vec{v_1}}{\vec{v_n}}{\vec{s_1}}{\vec{s_n}})\ansubst{\sigma}{}\real A\]


    \item[UnitCase] If the hypotheses are valid, then:
    \begin{itemize}
        \item $\sdom{\Gamma}\subseteq \FV{\vec{t}}\subseteq \dom{\Gamma}$
        \item For every $\sigma_\Gamma\in\sem{\Gamma}$, $\vec{t}\ansubst{\sigma_\Gamma}{}\real\sharp\genbasis{\vec{v_i}}{i=1}{n}$
        \item For every $i$, $\sdom{\Delta}\subseteq \FV{\vec{s_i}}\subseteq \dom{\Delta}$
        \item For every $i\in\{0,\dotsb ,n\}, \sigma_\Delta\in\sem{\Delta}$, $\vec{s_i}\ansubst{\sigma_\Delta}{}\real A$
    \end{itemize}
    
    From this we can conclude that:
    
    \begin{itemize}
        \item $\sdom{\Gamma,\Delta}\subseteq \FV{\gencase{\vec{t}}{\vec v_1}{\vec v_n}{\vec{s_1}}{\vec{s_n}}}$
        \item $\FV{\gencase{\vec{t}}{\vec v_1}{\vec v_n}{\vec{s_1}}{\vec{s_n}}}\subseteq \dom{\Gamma,\Delta}$
    \end{itemize}
    
    Then, given $\sigma\in\sem{\Gamma,\Delta}$, we have that $\ansubst{\sigma}{}=\ansubst{\sigma_\Gamma}{}\ansubst{\sigma_\Delta}{}$ for some $\sigma_\Gamma\in\sem{\Gamma}$ and $\sigma_\Delta\in\sem{\Delta}$. Using the first hypothesis we have that, $\vec{t}\ansubst{\sigma_\Gamma}{}\real\sharp\genbasis{\vec{v_i}}{i=1}{n}$, then $\vec{t}\ansubst{\sigma_\Gamma}{}\eval e^{i\theta_1}\cdot\vec{u}=e^{i\theta_1}\cdot(\sum_{i=1}^{n}\beta_i \vec{v_i})$ where $\sum_{i=1}^{n}|\beta_i|^2$. From the second hypothesis we have that $\vec{s_i}\ansubst{\sigma_\Delta}{}\eval e^{i\rho_i}\cdot\vec{u_i}\in\sem{A}$ for $i\in\{1,\dotsb ,n\}$ and $u_i\perp u_j$ if $i\neq j$. Therefore:

    \begin{align*}
        (&\gencase{\vec{t}}{\vec{v_1}}{\vec{v_n}}{\vec{s_1}}{\vec{s_n}})\ansubst{\sigma}{}\\ 
        &= (\gencase{\vec{t}}{\vec{v_1}}{\vec{v_n}}{\vec{s_1}}{\vec{s_n}})\ansubst{\sigma_\Gamma}{}\ansubst{\sigma_\Delta}{}\\
        &=(\sum_{i=1}^{n}\alpha_i \gencase{\vec{t}[\sigma_{\Gamma i}]}{\vec{v_1}}{\vec{v_n}}{\vec{s_1}}{\vec{s_n}})\ansubst{\sigma_\Delta}{} \\
        &\equiv (\gencase{\sum_{i=1}^{n} \alpha_i \vec{t}[\sigma_{\Gamma i}]}{\vec {v_1}}{\vec{v_n}}{\vec{s_1}}{\vec{s_n}})\ansubst{\sigma_\Delta}{}\\
        &=(\gencase{\vec{t}\ansubst{\sigma_\Gamma}{}}{\vec{v_1}}{\vec{v_n}}{\vec{s_1}}{\vec{s_n}})\ansubst{\sigma_\Delta}{}\\
        &\eval(\gencase{e^{i\theta_1}\cdot\vec{u}}{\vec{v}}{\vec{w}}{\vec{s_1}}{\vec{s_2}})\ansubst{\sigma_\Delta}{}\\
        &\evalone e^{i\theta_1}\cdot(\sum_{i=1}^{n}\beta_i s_i)\ansubst{\sigma_\Delta}{}\\
        &= e^{i\theta_1}\cdot(\sum_{j=1}^{n}\delta_j (\sum_{i=1}^{n}\beta_i \vec{s_i})[\sigma_{\Delta j}])\\
        &= e^{i\theta_1}\cdot(\sum_{j=1}^{n}\delta_j (\sum_{i=1}^{n}\beta_i \vec{s_i}[[\sigma_{\Delta j}]]))\\
        &\equiv e^{i\theta_1}\cdot(\sum_{i,j=1}^{n}\beta_i\delta_j\vec{s_i}[\sigma_{\Delta j}])\\
        &= e^{i\theta_1}\cdot(\sum_{i=1}^{n}\beta_i \vec{s_i}\ansubst{\sigma_\Delta}{})\\
        &\eval e^{i\theta_1}\cdot(\sum_{i=1}^{n}\beta_i e^{i\rho_i}\cdot\vec{u_i})
    \end{align*}
    
    It remains to be seen that: $\|\sum_{i=1}^{n}\beta_i e^{i\rho_i}\cdot\vec{u_i}\|=1$:
    \begin{align*}
        \|\sum_{i=1}^{n}\beta_i e^{i\rho_i}\cdot\vec{u_i}\| &= \scal{\sum_{i=1}^{n}\beta_i e^{i\rho_i}\cdot\vec{u_i}}{\sum_{i=1}^{n}\beta_i e^{i\rho_i}\cdot\vec{u_i}}\\
        &= \sum_{i,j=1}^{n}\overline{\beta_i e^{i\rho_i}}\beta_j e^{i\rho_j} \scal{\vec{u_i}}{\vec{u_j}}\\
        &= \sum_{i=1}^{n}\overline{\beta_i e^{i\rho_i}}\beta_i e^{i\rho_i} \scal{\vec{u_i}}{\vec{u_i}}\\
        &\qquad + \sum_{i,j=1; i\neq j}^{n}\overline{\beta_i e^{i\rho_i}}\beta_j e^{i\rho_j} \scal{\vec{u_i}}{\vec{u_j}}\\
        &= \sum_{i=1}^{n}|\beta_i|^2 |e^{i\rho_i}|^2  + 0\\
        &= \sum_{i=1}^{n}|\beta_i|^2 = 1
    \end{align*}

    Then we can conclude that $\sum_{i=1}^{n}\beta_i e^{i\rho_i}\vec{u_i}\in\sem{\sharp A}$ and finally:
    \[
        (\gencase{\vec{t}}{\vec{v_1}}{\vec{v_n}}{\vec{s_1}}{\vec{s_n}})\ansubst{\sigma}{}\real\sharp A
    \]

    \item[Sum] If the hypothesis is valid then for every $i$, $\sdom{\Gamma}\subseteq\FV{\vec{t_i}}\subseteq\dom{\Gamma}$.
    
    From this we can conclude that $\sdom{\Gamma}\subseteq\sum_{i=1}^{n}\alpha_i \vec{t_i}\subseteq\dom{\Gamma}$. Given $\sigma\in\sem{\Gamma}$, we have for every $i$, $\vec{t_i}\ansubst{\sigma}{}\eval e^{i\rho_i}\cdot\vec{v_i}$ where $\vec{v_i}\in\sem{A}$. Moreover, for every $i\neq j$, $\vec{v_i}\perp\vec{v_j}$ and $\sum_{i=1}^{n}|\alpha_i|^2=1$. Then:
    \begin{align*}
    (\sum_{i=1}^{n}\alpha_i\vec{t_i})\ansubst{\sigma}{} 
    &= \sum_{j=1}^{m}\beta_j(\sum_{i=1}^{n}\alpha_i \vec{t_i})[\sigma_j]\\
    &\equiv \sum_{i=1}^{n} \alpha_i \sum_{j=1}^{m} \beta_j \vec{t_i}[\sigma_j]\\
    &=\sum_{i=1}^{n} \alpha_i \vec{t_i}\ansubst{\sigma}{}\\
    &\eval \sum_{i=1}^{n} \alpha_i e^{i\rho_i} \vec{v_i}\\
    \end{align*}

    It remains to be seen that $\|\sum_{i=1}^{n} \alpha_i e^{i\rho_i} \vec{v_i}\|=1$. But:
    \begin{align*}
    &\|\sum_{i=1}^{n} \alpha_i e^{i\rho_i} \vec{v_i}\| \\
    &=\scal{\sum_{i=1}^{n} \alpha_i e^{i\rho_i}\vec{v_i}}{\sum_{i=1}^{n} \alpha_i e^{i\rho_i} \vec{v_i}}\\
    &= \sum_{i=i}^{n}\sum_{j=1}^{n} \overline{\alpha_i e^{i\rho_i}}\alpha_j e^{i\rho_j} \scal{\vec{v_i}}{\vec{v_j}}\\
    &=\sum_{i=1}^{n} \overline{\alpha_i e^{i\rho_i}}\alpha_i e^{i\rho_i} \scal{\vec{v_i}}{\vec{v_i}} + \sum_{\substack{i,j=1\\i\neq j}}^{n} \overline{\alpha_i e^{i\rho_i}}\alpha_j e^{i\rho_j} \scal{\vec{v_i}}{\vec{v_j}}\\
    &=\sum_{i=1}^{n}|\alpha_i|^2 |e^{i\rho_i}|^2+ 0\\
    &=\sum_{i=1}^{n}|\alpha_i|^2 = 1\\
    \end{align*}

    Then we can conclude that $\sum_{i=1}^{n}\alpha_i e^{i\rho_i}\vec{v_i}\in\sem{\sharp A}$ and finally $(\sum_{i=1}^{n}\alpha_i\vec{t_i})\ansubst{\sigma}{}\real\sharp A$.

    \item[Weak] Given $\sigma\in\sem{\Gamma,x_A:B}$, we observe that $\ansubst{\sigma}=\ansubst{\sigma_\Gamma}{}\ansubst{\vec{v}/x}{A}$ for some $\sigma_\Gamma\in\sem{\Gamma}$ and $\vec{v}\in\sem{B}$. Using the first hypothesis, we know that $\vec{t}\ansubst{\sigma_\Gamma}{}\eval e^{i\theta}\vec{w}$ where $\vec{w}\in\sem{B}$. Then we have:
    \[
    \vec{t}\ansubst{\sigma}{}=\vec{t}\ansubst{\sigma_\Gamma}{}\ansubst{\vec{v}/x}{A}\eval e^{i\theta}\vec{w}\ansubst{\vec{v}/x}{A}
    \]
    Since $\vec{v}\in\sem{A}$, $\vec{w}\ansubst{\vec{v}/x}{A}=\vec{w}[\vec{v}/x]=\vec{w}$ and $\vec{w}\in\sem{B}$, we can finally conclude that $\vec{t}\ansubst{\sigma}{}\real B$.

    \item[Contr] If the hypothesis is valid, we have that $\sdom{\Gamma, x_A:A, y_A:A}\subseteq\FV{\vec{t}}\subseteq\dom{\Gamma,x_A:A, y_A:A}$ and given $\sigma\in\sem{\Gamma,x_A:A, y_A:A}$, then $\vec{t}\ansubst{\sigma}{}\in\sem{B}$. Since we assume $\flat A$, we have that $\sdom{\Gamma,x_A:A, y_A:A}=\sdom{\Gamma,x_A:A}$. Therefore:
    
    \[
    \sdom{\Gamma,x_A:A}\subseteq\FV{\vec{t}}[x/y]\subseteq\dom{\Gamma,x_A:A}
    \]

    Given $\sigma\in\sem{\Gamma,x_A:A}$, we observe that $\ansubst{\sigma}{}=\ansubst{\vec{v}/x}{A}\ansubst{\sigma_\Gamma}{}$ with $\sigma_\Gamma\in\sem{\Gamma}$ and $\vec{v}\in\sem{A}$. Since $\vec{v}\in\sem{A}$, we know that $\vec{t}[\vec v/z] =\vec{t}\ansubst{\vec{v}/z}{A}$ for any variable $z$. Then we have:
    \begin{align*}
        \vec{t}[x/y]\ansubst{\sigma}{} &= \vec{t}[x/y]\ansubst{\vec{v}/x}{A}\ansubst{\sigma_\Gamma}{}\\
        &=\vec{t}[x/y][\vec{v}/x]\ansubst{\sigma_\Gamma}{}\\
        &=\vec{t}[\vec{v}/y][\vec{v}/x]\ansubst{\sigma_\Gamma}{}\\
        &=\vec{t}\ansubst{\vec{v}/y}{A}\ansubst{\vec{v}/x}{A}\ansubst{\sigma_\Gamma}{}    
    \end{align*}
    
    Since $\ansubst{\vec{v}/y}{A}\ansubst{\vec{v}/x}{A}\ansubst{\sigma}{}\in\sem{\Gamma,x_A:A,y_A:A}$, we get:
    \[\vec{t}\ansubst{\vec{v}/y}{}\ansubst{\vec{v}/x}{A}\ansubst{\sigma_\Gamma}{}\eval e^{i\theta}\vec{w}\in\sem{B}\]
    Then we can finally conclude that $\vec{t}[x/y]\ansubst{\sigma}{}\real B$.

    \item[Equiv] It follows from definition and the fact that the reduction commutes with the congruence relation.
    
    \item[GlobalPhase] It follows from the definition of type realizers.
    \end{description}
\end{proof}


\subsection{Discussion: Towards a specification system}

We found a somewhat interesting extreme when we drop the condition of bases to have size $n$, and we consider singletons. The resulting system replicates the term reduction inside the strict typing rules and forms a sort of specification system. In short, sequents take the following form: $x_1: \vec{v}_1, \dotsb, x_n: \vec{v}_n \vdash \vec t : \vec{w}$ (We omit the brackets in singleton sets). And so, we can read the previous sequent as: ``Every substitution $\sigma$ such that $\sigma(x_i)=\vec{v_i}$ validates that $\sigma(\vec{t})$, reduces to $\vec{w}$''. With this in mind, we can design inference rules that only use singleton sets as types. Let us take for example any non-abstraction value distribution $\vec{v}$, we describe a specification with tight typing in table \ref{tab:duplicationSpec}.

\begin{table*}
\scriptsize
\[  
\infer[\snam{App}]
{\TYP{}{(\lambda x^{\vec{v}}. (x, x))\ \vec{v}}{\basis{\vec{v}}\times\basis{\vec{v}}}}
{\infer[\snam{Lam}]
  {\TYP{}{\lambda x^{\vec{v}}. (x, x)}{\basis{\vec{v}} \Arr (\basis{\vec{v}}\times\basis{\vec{v}})}}
  {
    \infer[\snam{Contr}]
    {\TYP{x:\basis{\vec{v}}}{(x, x)}{\basis{\vec{v}}\times\basis{\vec{v}}}}
    {
      \infer[\snam{Pair}]
      {\TYP{x:\basis{\vec{v}}, y:\basis{\vec{v}}}{(x, y)}{\basis{\vec{v}}\times\basis{\vec{v}}}}
      {
        \infer[\snam{Ax}]{\TYP{x:\basis{\vec{v}}}{x}{\basis{\vec{v}}}}{}&
        \infer[\snam{Ax}]{\TYP{y:\basis{\vec{v}}}{y}{\basis{\vec{v}}}}{}
      }
    }
  } &
  \infer[\snam{Ax}]
  {\TYP{}{\vec{v}}{\basis{\vec{v}}}}
  {\vec{v}\in\{\vec{v}\}}
}
\]
\caption{Duplication of a vector $\vec{v}$}
\label{tab:duplicationSpec}
\end{table*}

This result is not surprising. When dealing with classic computation, regardless of the basis, if we restrict the possible inputs to only one option, we can statically determine the output without having to reduce the term.

However, this changes when entangled qubits are involved. Entangled quantum states are states where a qubit cannot be described precisely and independently of another qubit. So for example the state $\Bell=\frac{1}{\sqrt{2}}(\Pair{\ket{0}}{\ket{0}} + \Pair{\ket{1}}{\ket{1}})$ cannot be described as a pair of two separate values $\vec{v}$ and $\vec{w}$. We can however use the $\mathsf{let}$ construct to destroy the pair, and in that case, we will have to type two separate variables. The typing judgement is thus:

\begin{adjustbox}{width=\textwidth}
$$
\infer[\snam{LetTens}]{\TYP{}
{\LetP{x}{\B}{y}{\B}{\Bell}{\Pair{x}{y}}}{\sharp(\B\times\B)}}
{
    \infer[\snam{Sub}]
    {\TYP{}{\Bell}{\sharp(\B\times\B)}}
    {
        \infer[\snam{Basis}]{\TYP{}{\Bell}{\basis{\Bell}}}{\Bell\in\{\Bell\}} &
        \basis{\Bell}\leq \sharp(\B\times\B)
    }&
    \infer[\snam{Sub}]
    {\TYP{x:\sharp\B,y:\sharp\B}{\Pair{x}{y}}{\B\otimes\B}}
    {
        \infer[\snam{Pair}]
        {\TYP{x:\sharp\B,y:\sharp\B}{\Pair{x}{y}}{\sharp\B\times\sharp\B}}
        {
            \infer[\snam{Ax}]{\TYP{x:\sharp\B}{x}{\sharp\B}}{} &
            \infer[\snam{Ax}]{\TYP{y:\sharp\B}{y}{\sharp\B}}{}
        }&
        \sharp\B\times\sharp\B\leq\sharp(\B\times\B)
    }
}
$$
\end{adjustbox}

Despite knowing exactly which state we are referring to, there is an inherent loss of precision, and this itself is reflected in the type we are able to give. While this might seem as a limitation of the language, it is the desired outcome. This is due to the physical nature of the state, which cannot be described with two independent variables.

This mechanism outlines a possible implementation of a specification system similar to Hoare logic via realizability. It could be useful to reason about correctness in quantum programs. At the same time, we identify some limitations when analyzing terms which deal with entangled states. We let this development for future research.
