\section{Introduction}

% Motivation
We previously presented the impossibility theorems which stated that is physically impossible to copy or delete a qubit. There is however, a subtlety in these impossibility theorems. Arbitrary qubits cannot be copied, but it is indeed possible to do so with known qubits. This implies that qubits with known values
behave as classical data and can be treated accordingly. Moreover, it suffices to know the basis to which a qubit belongs in order to copy and delete it. This is a known fact in quantum information theory which underlies a number of quantum algorithms. 

In most quantum programming languages, qubits are interpreted in a canonical basis (often called the computational basis); see, for instance {(\color{red}CITAR EJEMPLOS)}. In this fashion, classical bits are represented by the basis vectors, and qubits as norm-1 linear combinations of bits. We are allowed to copy and delete classical bits freely, while such operations on arbitrary qubits remain restricted.

% Work in this chapter
In this chapter we will introduce a quantum lambda calculus in the quantum-data / quantum-control paradigm. It uses as starting point the calculus defined in \cite{DiazcaroGuillermoMiquelValironLICS19}, which was introduced using a realizability technique. In the same manner, our aim is to follow this workflow to extract a type system able to track bases throughout the programs. This should allow us to treat qubit in known bases classically, while still handling unknown qubits linearly.

% How are planning to do this (Realizability technique)

Realizability is a technique for extracting type systems from the operational semantics of a calculus, resulting in a system in which safety properties hold by construction.

The steps to define a programming language using this technique are as follows. First, define a calculus equipped with a deterministic evaluation strategy. Second, define types as sets of closed values in the language, optionally introducing operations to build more complex types. Third, define the typing judgement $\Gamma \vdash t : A$, where $\Gamma$ is a context of typed variables, $t$ a term in the calculus, and $A$ a type, as the property that for every valid substitution $\theta$ of $\Gamma$, the term $\theta(t)$ reduces to a value in $A$, i.e., $\theta(t) \twoheadrightarrow v \in A$.

In this setting, each typing rule corresponds to a provable theorem. For instance, if $\Gamma \vdash t : A$ implies $\Delta \vdash r : B$, then the following rule is derivable:
\[
  \infer{\Delta\vdash r:B}{\Gamma\vdash t:A}
\]

% Structure of the chapter
The structure of the chapter is as follows: In section \ref{sec:calculus}, we define the syntax for the calculus. Then, in \ref{sec:reduction} we detail the reduction system. We define the type algebra and prove a set of valid typing rules in section \ref{sec:model}. With the calculus fully defined, we showcase a few examples in section \ref{sec:examples}. We give closing remarks and discuss future work in \ref{sec:conclusion}.
